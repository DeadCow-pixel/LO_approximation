\documentclass[% 
reprint,
superscriptaddress,
%groupedaddress,
%unsortedaddress,
%runinaddress,
%frontmatterverbose, 
%preprint,
%showpacs,preprintnumbers,
%nofootinbib,
%nobibnotes,
%bibnotes,
 amsmath,amssymb,amsfonts,
 aps,
 pra,
 longbibliography
%prb,
%rmp,
%prstab,
%prstper,
%floatfix,
]{revtex4-2}

\newcommand{\ket}[1]{\ensuremath{|{#1}\rangle}}
\newcommand{\bra}[1]{\ensuremath{\langle{#1}|}}
\newcommand{\sca}[2]{\ensuremath{\bigl({#1}\cdot{#2}\bigr)}}
\newcommand{\avr}[1]{\ensuremath{\langle{#1}\rangle}}

\newcommand{\cnj}[1]{{#1}^{\ast}}
\newcommand{\hcnj}[1]{{#1}^{\dagger}}
\newcommand{\tcnj}[1]{{#1}^{T}}

%Differential operators

\newcommand{\prt}[1]{\partial_{#1}}
\newcommand{\pdrs}[1]{\partial_{#1}}
\newcommand{\pdr}[2]{\frac{\partial #1}{\partial #2}}
\newcommand{\drf}[2]{\frac{\dd #1}{\dd #2}}
\newcommand{\vdr}[2]{\dfrac{\delta #1}{\delta #2}}

\newcommand{\ddiv}{\mathop{\rm div}\nolimits}
 \newcommand{\rrot}{\mathop{\rm rot}\nolimits}
 \newcommand{\grad}{\mathop{\rm grad}\nolimits}
\newcommand{\bnbl}{\boldsymbol{\nabla}}

%Functions

\newcommand{\diag}{\mathop{\rm diag}\nolimits}
\newcommand{\sign}{\mathop{\rm sign}\nolimits}
\renewcommand{\Re}{\mathop{\rm Re}\nolimits}
\renewcommand{\Im}{\mathop{\rm Im}\nolimits}
\newcommand{\Tr}{\mathop{\rm Tr}\nolimits}
\newcommand{\ad}{\mathop{\rm ad}\nolimits}
\renewcommand{\ker}{\mathop{\rm Ker}\nolimits}

\newcommand{\cn}{\mathop{\rm cn}\nolimits}
\newcommand{\sn}{\mathop{\rm sn}\nolimits}
\newcommand{\dn}{\mathop{\rm dn}\nolimits}

%Units

\newcommand{\mum}{$\mu$m}
\newcommand{\dega}{$^\circ$}
\newcommand{\degc}{$^\circ$C}
\newcommand{\myarrow}[1]{\ensuremath{\xrightarrow{{#1}^\circ\mathrm{C}}}}

%Symbols

 \newcommand{\bs}[1]{\boldsymbol{#1}}
 \newcommand{\vc}[1]{\mathbf{#1}}
 \newcommand{\mvc}[1]{\mathbf{#1}}
 \newcommand{\uvc}[1]{\hat{\mathbf{#1}}}
 \newcommand{\ubs}[1]{\hat{\boldsymbol{#1}}}
 \newcommand{\ind}[1]{\mathrm{#1}}
\newcommand{\oprt}[1]{\ensuremath{\widehat{\mathcal{#1}}}}

%% 1. Math
\newcommand{\prob}{\mathsf{P}}
\newcommand{\dd}{\mathrm{d}}
\newcommand{\DD}{\ensuremath{\mathcal{D}}}
\newcommand{\JJ}{\ensuremath{\mathcal{J}}}
 \newcommand{\e}{\mathrm{e}}
 \newcommand{\ee}{\mathrm{e}}
\newcommand{\sts}{\ensuremath{\mathcal{Z}}}

\newcommand{\eff}{\mathrm{eff}}
\newcommand{\mx}{\mathrm{max}}
\newcommand{\mn}{\mathrm{min}}

\usepackage[english]{babel}
\usepackage[utf8x]{inputenc}
\usepackage[T1]{fontenc}


%% Useful packages
\usepackage{amsmath}
%\usepackage[colorinlistoftodos]{todonotes}
%\usepackage[colorlinks=true, allcolors=blue]{hyperref}
\usepackage{amssymb,amsmath,amsthm,mathtools,mathrsfs}
\usepackage{graphicx}% Include figure files
\usepackage{multirow}% Include figure files
\usepackage{float}% Include figure files
\usepackage{subcaption}% Include figure files


%\graphicspath{{figs/}}
\renewcommand{\floatpagefraction}{0.8}

%\selectlanguage{english}

\usepackage{soul,color}
\begin{document}
\DeclareGraphicsExtensions{.png,.pdf}
\title{
  Asymmetry effects in homodyne-like measurements: Positive operator-valued measures and quantum key distribution
}

\author{A. S.~Naumchik}
\email[Email address: ]{naumchik95@gmail.com}
\affiliation{ITMO University, Kronverksky Pr. 49, Saint Petersburg  197101, Russia}


\author{Roman~K.~Goncharov}
\email[Email address: ]{toloroloe@gmail.com}
\affiliation{ITMO University, Kronverksky Pr. 49, Saint Petersburg  197101, Russia}


\author{Alexei~D.~Kiselev}
\email[Email address: ]{alexei.d.kiselev@gmail.com}
\affiliation{Laboratory of Quantum Processes and Measurements, ITMO
  University, Kadetskaya Line 3b, Saint Petersburg 199034, Russia} 
\affiliation{Leading Research Center "National Center for Quantum Internet", ITMO University,
  Birzhevaya Line 16, Saint Petersburg 199034, Russia}

\date{\today}

\begin{abstract}
  We study applicability of
  the Gaussian approximation describing
  photon count statistics for both the homodyne and the double homodyne
  measurements in the presence of asymmetry effects when
  the beam splitters are unbalanced and the quantum efficiencies of
  the photodetectors are not identical.
We also use the Gaussian approximation to construct
the positive operator-valued measure (POVM) that takes into account the asymmetry effects. The results are applied to calculate the secure key rate of the GG02 protocol under an untrusted noise model. It is found that asymmetry is detrimental in this case.
Moreover, we found that the double homodyne POVM is not uniquely defined, because the squeezing parameter, which governs the POVM, is constrained within an interval determined by the asymmetry parameters of the measurement scheme. Consequently, the Holevo information, which depends on this POVM, also varies with the squeezing parameter within this range.
%Moreover, when the Skellam distribution is approximated using the asymptotic expansions of modified Bessel functions for large argument, the asymmetry effects lead to ill-posed Gaussian POVM.
\end{abstract}
 
 \maketitle

%%%%%%%%%%%%%%
\section{Introduction}
\label{sec:intro}
%%%%%%%%%%%%%%

Homodyne detection is a fundamental technique in quantum optics for measuring quadrature components (amplitude and phase) of light fields and plays a central role in continuous-variable quantum key distribution (CV-QKD) systems~\cite{PhysRevLett.88.057902,PhysRevLett.93.170504,RevModPhys.84.621,e17096072,opt3040030,Zhang:apr:2024,Gaidash:pra:2024}. In a typical setup, a weak quantum signal is combined with a classical local oscillator (LO) at a beam splitter, and the two outputs are detected by photodiodes; the difference photocurrent yields information about the signal quadratures~\cite{Vogel:pra:1993,Kelley:pr:1964}.

Most theoretical treatments assume the strong-local-oscillator (LO) approximation, where the LO is much more intense than the signal~\cite{Vogel:bk:2006}. This simplification improves analytical tractability and measurement precision but may also amplify classical LO noise and mask subtle quantum effects such as squeezing. In the weak-LO regime, where the LO and signal amplitudes are comparable, phase-sensitive quantum effects become more visible~\cite{PhysRevA.51.4160}; however, operation in this regime requires highly efficient and temporally stable detectors to compensate for low optical power~\cite{PhysRevA.51.4160,Schumaker:84}. Various measurement schemes, including homodyne intensity correlations with two beam splitters and detectors or cross-correlation using a single unbalanced beam splitter, are useful when the overall detection efficiency is limited~\cite{PhysRevA.51.4160}.

Approaches such as unbalanced homodyne detection, where only one output port of the beam splitter is measured, have been employed for quantum-state reconstruction based on positive operator-valued measures (POVMs) associated with $s$-parametrized quasiprobability distributions ($s<1$)~\cite{PhysRevA.53.4528}. These schemes suffer from high noise sensitivity and current limitations in photon-number resolution, although detector arrays have been proposed to mitigate these drawbacks~\cite{PhysRevA.101.031801,PhysRevA.85.023820,PhysRevA.92.053835}.

Another important approach is double-homodyne (eight-port) detection, which yields simultaneous outcomes for both conjugate quadratures and allows direct reconstruction of the Husimi $Q$-function~\cite{Richter:98,PhysRevA.101.031801}. At finite LO intensities, this scheme bridges the classical and quantum regimes; in the weak-LO limit, it reduces to photon-number measurements~\cite{Cives_Esclop_2000}. In the context of CV-QKD protocols, this method also facilitates the symmetrization procedure, since the measurement retains complete information about both quadratures~\cite{PhysRevLett.118.200501}.

Practical implementations are affected by nonideal components and asymmetries that must be modeled explicitly. Imperfections such as unbalanced beam splitters, unequal detector efficiencies, finite photon-number resolution, and detector dead times introduce excess noise that degrades measurement fidelity and compromises CV-QKD security~\cite{PhysRevA.53.4528,len2022realistic,yeremenko2024realistic,reutov2021photon,hajomer2025finite,ruiz2023effects,Wang:23}. 
% Although the ideal POVMs for projections onto quadrature or coherent states are well known~\cite{Vogel:bk:2006,Richter:98}, a detailed explicit quantum treatment of asymmetry effects -- namely, the beam-splitter imbalance and the detector-efficiency mismatch -- has not yet been carried out in the literature~\cite{PhysRevA.85.023820,PhysRevA.92.053835}. In this work, we therefore focus on these two imperfections and refer to them collectively as asymmetry effects.

Although the ideal POVMs for projections onto quadrature or coherent states are well known~\cite{Vogel:bk:2006,Richter:98}, a comprehensive quantum-optical treatment of detector asymmetry -- explicitly incorporating the beam-splitter imbalance and detector-efficiency mismatch into the measurement operators -- has not yet been developed~\cite{PhysRevA.85.023820,PhysRevA.92.053835}. 
Existing complementary studies, such as Ref.~\cite{Bartlett:25}, have investigated the practical influence of detection imbalance on phase-reference estimation and key-rate performance in CV-QKD systems and illustrated its impact experimentally using Wigner-function reconstructions. 
However, these analyses remain semi-classical and do not address the modification of the measurement process itself within the quantum-optical formalism. 
In this work, we focus on these imperfections at the level of the quantum measurement and refer to them collectively as asymmetry effects.

In practice, these imperfections are often modeled as additive technical noise (electronic noise, dark counts, finite bandwidth), but asymmetry introduces structured excess noise that must be accounted for to produce reliable security estimates~\cite{Wang:23,ruiz2023effects,usenko2016trusted,Bartlett:25}. Such imperfections may also open side channels exploitable by an adversary (e.g., wavelength-dependent or detector-blinding attacks), so countermeasures such as spectral filtering, detector balancing, and careful calibration are essential~\cite{huang2012wavelength,huang2014quantum,qin2018homodyne,qin2016quantum,Wang:23}. Explicitly accounting for measurement asymmetry is therefore an essential part of this defensive toolbox.


%%%%%%%%%%%%%%%%%%%%%%%%%%%%%%%%
Our analysis focuses on an asymmetrical detection scenario: an unbalanced beam splitter combined with unequal (and non-unity) quantum efficiencies of the photodetectors, in contrast to the conventional symmetrical case of a balanced beam splitter and identical detector efficiencies. To derive tractable expressions, we model the photocount statistics by approximating Poisson distributions with Gaussian ones (the Gaussian approximation) and further simplify using the strong-LO approximation. From these approximations, we construct the corresponding POVMs and numerically assess the accuracy of the resulting expressions.

We then apply the developed formalism to a homodyne-based two-quadrature scheme (double homodyne, often referred to as heterodyne in the QKD literature~\cite{Pirandola:20,opt3040030,Zhang:apr:2024}) to demonstrate applicability beyond single-quadrature detection. We find that the asymmetrical double-homodyne POVM requires an extension to the set of squeezed coherent states. We then use the obtained asymmetrical POVMs to compute the CV-QKD asymptotic secret fraction, thereby quantifying the impact of measurement asymmetry on protocol security.

% We also show that the commonly used Skellam-distribution approximation, based on the asymptotic expansion of the modified Bessel function of the first kind, is not valid in the asymmetrical regime.

The paper is organized as follows. In Sec.~\ref{sec:homodyne}, we derive the statistical distribution of difference photon counts in the Gaussian approximation, from which we obtain the respective POVM. In Sec.~\ref{sec:double-homodyne}, we analyze the double-homodyne scheme analogously to the previous section and show that the resulting POVM is not well defined for all asymmetry parameters, requiring generalization. In Sec.~\ref{sec:gen-POVM}, we generalize the double-homodyne POVM to the set of squeezed coherent states. In Sec.~\ref{sec:protocol}, we apply our results for homodyne and double-homodyne detection to calculate the asymptotic secret fraction for the GG02 CV-QKD protocol. Finally, in Sec.~\ref{sec:conclusion}, we summarize the main results and outline directions for future research.

\begin{figure}
    \centering
    \includegraphics[width=0.75\linewidth]{pics/schemes/homodyne.pdf}
    \caption{Scheme of a homodyne receiver: S is the source of the
      signal mode with the annihilation operator $\hat{a}$,
      LO is the source of the reference mode (local oscillator) with the annihilation
      operator $\hat{a}_{L}$, and BS is the beam splitter with
      the amplitude transmission and reflection coefficients
      $t=\cos\theta$ and $r=\sin\theta$, respectively; photodetectors $D_1$ and $D_2$ have quantum efficiencies
     $\eta_{1}$ and $\eta_{2}$, and $\mu\equiv m_1-m_2$ is the photon count
      difference.}
    \label{fig:homodyne}
\end{figure}

%%%%%%%%%%%%%%%%%%%%%%%%%%%%
\section{Homodyne detection}
\label{sec:homodyne} 
%%%%%%%%%%%%%%%%%%%%%%%%%%%


We begin with brief discussion
of the homodyne measurement setup
schematically depicted in Fig.~\ref{fig:homodyne}.
To this end, we assume that the beam spitter is unbalanced
and its scattering matrix is chosen to be a real-valued
rotation matrix with the transmission and reflection amplitudes,
$t$ and $r$, given by
\begin{align}
  \label{eq:BS-amplitudes}
  t=\cos\theta\equiv C,
  \quad
  r=\sin\theta\equiv S.
\end{align}
Then the input coherent states
of the signal mode and the local oscillator 
are transformed into the output coherent states
as follows
\begin{align}
  \label{eq:BS}
  &
    \ket{\alpha,\alpha_L}\mapsto\ket{\alpha_1,\alpha_2},
  \\
  &
    \label{eq:amplitudes}
    \alpha_1=C\alpha+S\alpha_L,\quad
    \alpha_2=-S\alpha+C\alpha_L,
\end{align}
so that
the joint probability of $m_1$ and $m_2$ photon counts
for the photodetectors $D_1$ and $D_2$
can be computed from the well-known Kelley-Kleiner formula~\cite{Kelley:pr:1964}
(see also Ref.~\cite{Vogel:pra:1993}):
\begin{align}
  &
    \label{eq:1}
    \prob(m_1,m_2)=
    \bra{\alpha_1,\alpha_2}:\prod_{l=1}^{2}\frac{(\eta_l\hat{n}_l)^{m_l}e^{-\eta_l\hat{n}_l}}{m_l!}:\ket{\alpha_1,\alpha_2}
    \notag
  \\
  &
 =\prod_{l=1}^{2}\frac{(\eta_l|\alpha_l|^2)^{m_l}}{m_l!}e^{-\eta_l|\alpha_l|^2}    
\end{align}
where
$:\ldots:$ stands for normal ordering,
index $l\in\{1,2\}$ labels output ports of the beam splitter,
$\hat{n}_l=\hcnj{\hat{a}}_l\hat{a}_l$ is the photon number operator,
$m_l$ is the number of photon counts,
$\eta_l$ is the quantum efficiency of the detector $D_l$.

% and
% $|\alpha_l|$ are the transformed amplitudes, given by
% \begin{align}
%   \label{eq:amplitudes1}
%   &
%         \cos \theta\equiv C,\quad 
%                       \sin \theta\equiv S,
% \notag
%   \\
%   &
%     \alpha_1=C\alpha+S\alpha_L,\quad
%     \alpha_2=-S\alpha+C\alpha_L,
% \end{align}
% with the
We can now introduce the photon count difference
\begin{equation}
  \label{eq:dleta-m}
    \mu = m_1-m_2
  \end{equation}
  so that its statistical distribution %of the photon count difference
  can be written in
  the form of a product of the two Poisson distributions as follows 
\begin{align}
  &
\label{eq:poisson}  
    \prob(\mu) =\sum_{m_2=\max(0,-\mu)}^{\infty}
    \frac{(\eta_1|\alpha_1|^2)^{\mu+m_2}}{(\mu+m_2)!}e^{-\eta_1|\alpha_1|^2}
    \notag
  \\
  &
    \times
\frac{(\eta_2|\alpha_2|^2)^{m_2}}{m_2!}e^{-\eta_2|\alpha_2|^2}.
\end{align}
It is well known that, by performing summation over $m_2$,
the probabilty $\prob(\mu)$ reduces to
the Skellam distribution given by~\cite{skellam1946frequency}
\begin{align}
  &
  \label{eq:accurate}
  \prob(\mu)=e^{-\eta_1|\alpha_1|^2}e^{-\eta_2|\alpha_2|^2}
    \Biggl(\frac{\eta_1|\alpha_1|^2}{\eta_2|\alpha_2|^2}\Biggr)^{\mu/2}
    \notag
  \\
  &
    \times
I_{\mu}\bigl(2\sqrt{\eta_1\eta_2|\alpha_1|^2|\alpha_2|^2}\bigr),
\end{align}
where $I_k(z)$ is the modified Bessel function of the first kind~\cite{NIST:hndbk:2010}.

An important point is that,
at sufficiently large
$|\alpha_1|$ and $|\alpha_2|$,
Poisson distributions that enter Eq.~\eqref{eq:1}
can be approximated using
the probability density functions of the normal distributions
with mean and variance both equal to the mean of
the corresponding Poisson distribution, $\lambda_i=\eta_i |\alpha_i|^2$.
Then, in the continuum limit where
summation in Eq.~\eqref{eq:poisson} is replaced with integration,
the Skellam distribution~\eqref{eq:accurate}
can be approximated assuming
that the amplitude of the local oscillator, $|\alpha_L|$, is large
(the strong-LO approximation) and
we can apply the convolution formula
for Gaussian probability densities
\begin{align}
  &
  \label{eq:convolution}
  \int G(x_1-x_2;\sigma_1)G(x_2;\sigma_2)\dd
    x_2=G(x_1;\sigma_1+\sigma_2),
  \\
  &
    \label{eq:G-notation}
    G(x;\sigma)\equiv\frac{1}{\sqrt{2\pi\sigma}}\exp\Bigl(-\frac{x^2}{2\sigma}\Bigr).
\end{align}

The above procedure
immediately leads to the Gaussian approximation
of the form:
\begin{align}
  &
  \label{eq:Gaussian}
    \prob_G(\mu)=G(\mu-\mu_G;\sigma_G),
    \\
  &
    \label{eq:sigma_G}
    \sigma_G=\eta_1|\alpha_1|^2+\eta_2|\alpha_2|^2
    \approx
    (\eta_1S^2+\eta_2C^2) |\alpha_L|^2,
  \\
  &
    \label{eq:mu_G}
    \mu_G=\eta_1|\alpha_1|^2-\eta_2|\alpha_2|^2
    \approx
    (\eta_1S^2-\eta_2C^2) |\alpha_L|^2
    \notag
  \\
  &
    +CS (\eta_1+\eta_2) |\alpha_L| \avr{\hat{x}_\phi}
\end{align}
where
\begin{align}
  &
  \label{eq:avr-x-phi}
  \avr{\hat{x}_\phi}\equiv \bra{\alpha}\hat{x}_\phi\ket{\alpha}={2\Re\alpha e^{-i\phi}},
  \quad \phi=\arg\alpha_L
\end{align}
is the average of
the phase-rotated quadrature operator of the signal mode
given by
\begin{equation}
\hat{x}_\phi=\hat{a}e^{-i\phi}+\hat{a}^\dag e^{i\phi}.
    \label{eq:quad-op}
\end{equation}

Alternatively, the probability~\eqref{eq:Gaussian}
can be rewritten in the form
\begin{equation}
{\prob}_G(x)=\frac{1}{\sqrt{2\pi\sigma_G}}
    \exp \biggl\{-\frac{(x-\avr{\hat{x}_\phi})^2}{2\sigma_x}\biggr\},
    \label{eq:Pgood-w-def}
\end{equation}
where
$x$ is the quadrature variable given by
\begin{align}
  \label{eq:x-def}
  x\equiv\frac{\mu}{(\eta_1+\eta_2)CS|\alpha_L|}-\frac{\eta_1S^2-\eta_2C^2}{\left(\eta_1+\eta_2\right)CS}|\alpha_L|
\end{align}
and $\sigma_x$ is the quadrature variance
\begin{align}
  &
        \label{eq:sigma-x}
    \sigma_x\equiv
    \frac{\eta_1S^2+\eta_2C^2}{
    \left[(\eta_1+\eta_2)CS\right]^2
    }.
\end{align}
Note that it is rather straightforward to minimize
the variance~\eqref{eq:sigma-x} with respect to
the transmittance, $C^2$, and
deduce inequality
\begin{align}
  \label{eq:sigma-x-min}
  \sigma_x\ge \sigma^{(\min)}_{x}=\Biggl(
\frac{\sqrt{\eta_1}+\sqrt{\eta_2}}{\eta_1+\eta_2}
  \Biggr)^2\ge 1,
\end{align}
where $\sigma_x$ reaches its minimum value
$\sigma^{(\min)}_{x}$ at the beam splitter transmittance:
$C^2=\cos^2\theta_{\min}=\sqrt{\eta_1}/(\sqrt{\eta_1}+\sqrt{\eta_2})$.




Our next step is to construct
the positive operator-valued measure
(POVM) based on the Gaussian approximation $\prob_G$.
To this end, note that the probability~\eqref{eq:Pgood-w-def}
%\eqref{eq:Pgood}
is the expectation value of the POVM in the coherent state
given by
\begin{equation}
  \label{eq:PG-as-avr}
    \prob_G=\langle\alpha|\hat{\Pi}_G|\alpha\rangle.
\end{equation}
In the case of
the perfectly symmetric homodyne measurement with
$\eta_1=\eta_2=1$ and $C=S=1/\sqrt{2}$,
the average~\eqref{eq:PG-as-avr}
takes the form
\begin{equation}
    \label{eq:P_0}
    \prob_G^{(0)}=
    \frac{1}{|\alpha_L|} Q_{x,\phi}(\alpha), 
  \end{equation}
  where
  $x=\mu/|\alpha_L|$
  and $Q_{x,\phi}(\alpha)$ is the Husimi $Q$ distribution
for the eigenstate of the phase-rotated quadrature
operator~\eqref{eq:quad-op},
$\ket{x,\phi}$, given by
(see, e.g., the textbook\cite{Vogel:bk:2006})
\begin{align}
  &
    \label{eq:Q-x}
  Q_{x,\phi}(\alpha)=
    |\langle \alpha|x , \phi \rangle |^2 =G(x-\avr{\hat{x}}_\phi;1).
%    \frac{1}{\sqrt{2\pi}}\exp\left[-\frac{1}{2}\left(x  - \langle \hat{x}_\phi\rangle\right)^2\right]. 
\end{align}
Thus, we are led to the well-known result that POVM describing sharp homodyne measurements
in the Gaussian approximation is proportional to
a projector onto $|x,\phi\rangle$:
\begin{align}
  &
  \label{eq:POVM-P0}
    \hat{\Pi}_G^{(0)}=\frac{1}{|\alpha_L|}|x,\phi\rangle\langle x,\phi|,
\end{align}

In  a more general asymmetric case with $\eta_1\ne \eta_2$
and $C\ne S$,
the Gaussian-shaped probability $\prob_G$ can be represented
as a Gaussian superposition written as
a convolution of $P_{G}^{(0)}$ and a Gaussian function $G(x,\sigma_N)$.
By using the convolution identity~\eqref{eq:convolution},
we have
\begin{align}
  &
  \label{eq:theform}
  \prob_G(x)=\sqrt{\frac{\sigma_x}{\sigma_G}} \int G(x-x';\sigma_N)\prob_G^{(0)}(x')\dd x',
  \\
  &
  \label{eq:sgm_N}
     \sigma_N=\sigma_x-1\ge 0,  
\end{align}
where non-negativity of the variance $\sigma_N$
stems from Eq.~\eqref{eq:sigma-x-min}.
  This result immediately gives
  a general formula for the Gaussian approximation POVM 
\begin{align}
  &
\label{eq:homodyne-povm}    
    \hat{\Pi}_G=\frac{1}{(\eta_1+\eta_2)CS|\alpha_L|}
    \notag
  \\
  &
    \times
    \int \dd x' G(x-x'; \sigma_N)|x',\phi\rangle\langle x',\phi|.
\end{align}
Note that the variance $\sigma_N$
describes the excess noise that takes into account asymmetry effects.
In the limiting case of perfect homodyne,
we have
\begin{equation}
    \lim_{\sigma_x \to 1}G(x;\sigma_N)=\lim_{\sigma_N \to 0}G(x;\sigma_N)=\delta(x),
\end{equation}
where $\delta(x)$ is the Dirac $\delta$-function,
which is the expected behavior for
Eq.~\eqref{eq:theform} to hold .
Therefore, the constructed POVM~\eqref{eq:homodyne-povm}
is well-defined for all possible parameters of the homodyne scheme.

From Eq.~\eqref{eq:Pgood-w-def} we are also able to construct POVMs~\eqref{eq:homodyne-povm} covariance matrix:
\begin{align}
  \label{eq:Pi_h}
  &\Sigma^{\ind{H}}=\lim_{z\to0}R_\phi^{\ind{T}}\left[\ind{diag}(z,z^{-1})+\sigma_N\mathbb{I}\right]R_\phi\notag\\ &\equiv\Sigma_0^{\ind{H}}+\Sigma_N^{\ind{H}},
\end{align}
where $\Sigma_0^{\ind{H}}$ and $\Sigma_N^{\ind{H}}$ denote the covariance matrices of the ideal measurement and the associated excess noise, respectively, $\mathbb{I}=\ind{diag}(1,1)$, and $R_\phi$ is the rotation matrix by angle $\phi$.

% \textit{The ability of the homodyne scheme to reconstruct the quadrature of the state is well known
% (see e.g. Ref.~\cite{liu2021homodyne}), which is showcased by Eq.~\eqref{eq:homodyne-povm}.
% }
  




\begin{figure*}
    \centering
    \begin{subfigure}[]{.45\textwidth}
\includegraphics[width=\linewidth]{pics/homodyne/coherent_distributions.pdf}
\caption[]{$\ket{\psi}=\ket{\alpha},\:$$\alpha=0.5$}
\label{fig:dist_alp}
        \end{subfigure}
        \hfill
        \begin{subfigure}[]{.45\textwidth}
 \includegraphics[width=\linewidth]{pics/homodyne/fock_distributions.pdf}
\caption[]{$\ket{\psi}=\ket{n},\:n=1$}
\label{fig:dist_fock}
\end{subfigure}
\caption{Exact (circle dots) and approximate (solid lines with markers) statistical distributions of photon count difference
  for the signal mode prepared in (a)~the coherent state and
  in (b)~the single photon Fock state computed for
  for different efficiencies  at $|\alpha_L|=5$
  and balanced beamspliiter.
}
\label{fig:dist-H}
\end{figure*}



The exact and approximate analytical results for photon count difference statistical distributions, given by Eq.~\eqref{eq:accurate} and Eq.~\eqref{eq:Gaussian} respectively, are valid
for the case where the LO and signal modes are both
in the coherent states. In the more general case when the quantum state of the signal mode
is $\ket{\psi}$,
the probability distributions can be evaluated
using the relations
\begin{align}
  &
\label{eq:average-P-func}
    \prob(\mu;\ket{\psi})=\int  P_{\ket{\psi}}(\alpha)\prob(\mu;\alpha)\dd^2\alpha,
    \notag
  \\
  &
    \prob_G(x;\ket{\psi})=\bra{\psi}\hat{\Pi}_G\ket{\psi},
\end{align}
where $P_{\ket{\psi}}(\alpha)$ is the Glauber $P$ function of the quantum state
$\ket{\psi}$.
In Fig.~\ref{fig:dist-H},
we show the results computed for the single-photon Fock states
obtained utilizing
the well-known expression for
the $P$-function of Fock states $\ket{\psi}=\ket{n}$ given by
%\cite{vogel2006quantum}
\begin{equation}
\label{eq:P-function_fock}
P_{\ket{n}}(\alpha)=
\frac{e^{|\alpha|^2}}{n!}
\Bigl(
\frac{\partial^2}{\partial\alpha\partial\cnj{\alpha}}
\Bigr)^n
\delta^{2}(\alpha),
\end{equation}
where $\delta^{2}(\alpha)=\delta(\Re\alpha)\delta(\Im\alpha)$.

Figure~\ref{fig:dist-H} displays
the photocount difference probabilities
computed from
the exact and Gaussian probability distributions
for the balanced beam splitter at different photodetector efficiencies
and signal mode input states.
Fig.~\ref{fig:dist_alp} shows that,
in agreement with Eq.~\eqref{eq:mu_G},
asymmetry in photodetection
results in the shift of the probability maximum.
Note that, at $\delta\theta=0$,
the photocount variance~\eqref{eq:sigma_G},
$\sigma_G=(\eta_1+\eta_2)|\alpha_L|^2/2$, and
the quadrature variance~\eqref{eq:sigma-x},
$\sigma_x=2/(\eta_1+\eta_2)$,
are both invariant under transposition of the photodetectors:
$\eta_{1}\leftrightarrow \eta_{2}$.

The distributions for the single-photon states
are depicted in Fig.~\ref{fig:dist_fock}
and, similar to the coherent state,
demonstrate the effect of asymmetry-induced shift.
Another noticeable effect is that
the probability minima between the central and side peaks
become less pronounced.

From Fig.~\ref{fig:dist-H} it becomes apparent that perfomance of Gaussian approximation worsens in presence of asymmetry, which we will quantify in Appendix~\ref{sec:appendix_numerical}.

Our concluding remark concerns an alternative method
to approximate Eq.~\eqref{eq:accurate}
with a Gaussian-shaped distribution
which is based on the asymptotic expansions of the modified Bessel functions.
In Appendix~\ref{sec:appendix_comparison} we show that, for the asymmetric homodyne scheme,
this method generally leads to ill-posed POVMs because
the corresponding quadrature variance
appears to be too small leading to negative
contribution of the excess noise. 
  

\begin{figure}
    \centering
    \includegraphics[width=0.9\linewidth]{pics/schemes/double_homodyne.pdf}
    \caption{Scheme of an eight port double homodyne receiver.
      S is the source of the signal mode
      $\hat{a}$; LO is the source of the reference mode $\hat{a}_{L}$;
      BS$_S$ (BS$_L$) is the signal mode (local oscillator) beam splitter;
      $\frac{\pi}{2}$ is the quarter wave phase shifter;
BS$_i$ is the beam splitter of $i$th homodyne;
$D_{1,2}^{(i)}$ are the photodetectors of the $i$th homodyne;
and $\mu_i=m_1^{(i)}-m_2^{(i)}$ is the photon count difference registered by
the detectors of $i$th homodyne.
    }
    \label{fig:double-homodyne}
\end{figure}



%%%%%%%%%%%%%%%%%%%%%%%%%%%%%%%%%%%%%%%
% \section{Discussion}
% \label{sec:discussion}
%%%%%%%%%%%%%%%%%%%%%%%%%%%%%%%%%%%%%%%%

%\section{Double homodyne detection scheme}\label{sec-double-homodyne}

Consider the 8-port double homodyne scheme (Fig.~\ref{fig:double-homodyne}) \cite{lahti2010realistic}. It is known that this measurement scheme allows to reconstruct the $Q$-function of the signal state \cite{Richter:98}, yielding full information about the signal state, which may be used in CV-QKD protocols. It should be noted that the restoration of the state complex amplitude entirely serves as the basis for composable security proofs~\cite{PhysRevLett.93.170504,PhysRevLett.118.200501,pirandola2024improvedcomposablekeyrates,pascualgarcía2024improvedfinitesizekeyrates}. Analogous to Eq.~\eqref{eq:accurate}, the statistical distribution of photon count difference is written as:
\begin{multline}
P= \left(\frac{\eta_1|\alpha_1|^2}{\eta_4|\alpha_4|^2}\right)^{\frac{\delta m_1}{2}}
I_{\delta m_1}\left(2\sqrt{\eta_1\eta_4|\alpha_ 1|^2|\alpha_4|^2}\right)
e^{-\eta_1|\alpha_1|^2}
e^{-\eta_4|\alpha_4|^2}\times\\\times
\left(\frac{\eta_3|\alpha_3|^2}{\eta_2|\alpha_2|^2}\right)^{\frac{\delta m_2}{2}}
I_{\delta m_2}\left(2\sqrt{\eta_2\eta_3|\alpha_ 2|^2|\alpha_3|^2}\right)
e^{-\eta_2|\alpha_2|^2}
e^{-\eta_3|\alpha_3|^2}\label{eq:accurate-dh},
\end{multline}
where
\begin{align}
\begin{split}
    |\alpha_1|&=|
    C_1C_2\alpha
    -S_3S_2\alpha_L|,\\
    |\alpha_2|&=
    |S_1C_4\alpha
    -iC_3S_4\alpha_L|,\\
    |\alpha_3|&=|
     S_1S_4\alpha
    +iC_3C_4\alpha_L|,\\
    |\alpha_4|&=|
    C_1S_2\alpha+
    S_3C_2\alpha_L|.
\end{split}
\end{align}

The approximation method used in the previous section yields the Gaussian approximation
\begin{multline}
P_G=\frac{\biggl[(\eta_1S_2^2+\eta_4C_2^2)(\eta_3C_4^2+\eta_2S_4^2)\biggr]^{-\frac{1}{2}}}{2\pi C_3S_3|\alpha_L|^2}\times\\\times \exp \biggl[-D_1\left(x_1+\frac{\Re\alpha\alpha_L^*}{|\alpha_L|}\right)^2\biggr]\exp \biggl[-D_2\biggl(x_2-\frac{\Im\alpha\alpha_L^*}{|\alpha_L|}\biggr)^2\biggr], \label{eq:Pgood-dh}
\end{multline}
where designations
\begin{align}
    \begin{split}
D_1&=\frac{2(C_1C_2S_2[\eta_1+\eta_4])^{2}}{\eta_1S_2^2+\eta_4C_2^2},\\
    D_2&=\frac{2(S_3S_4C_4[\eta_3+\eta_2])^2}{\eta_3C_4^2+\eta_2S_4^2},
    \end{split}\label{eq:dh-D}
\end{align}
\begin{align}
    \begin{split}
x_1&=\frac{\delta m_1}{2|\alpha_L|C_1C_2S_2S_3[\eta_1+\eta_4]}+\frac{-\eta_1S_2^2S_3^2+\eta_4S_3^2C_2^2}{2C_1C_2S_2S_3[\eta_1+\eta_4]}|\alpha_L|, \\
    x_2&=\frac{\delta m_2}{2|\alpha_L|S_1S_4C_3C_4[\eta_3+\eta_2]}+\frac{-\eta_3C_3^2C_4^2+\eta_2C_3^2S_4^2}{2S_1S_4C_3C_4[\eta_3+\eta_2]}|\alpha_L|
    .
    \end{split}\label{eq:dh-x}
\end{align}
have been introduced. Note that Eq.~\eqref{eq:Pgood-dh} is a product of two functions in the form of Eq.~\eqref{eq:Pgood} (much like Eq.~\eqref{eq:accurate-dh} is a product of two function in the form of Eq.~\eqref{eq:accurate}), meaning approximation quality study in previous section fully applies for this case. Indeed, the difference statistics of the double homodyne scheme is difference statistics of two separate homodyne schemes with quadrature displaced by $\frac{\pi}{2}$, normalized to unity (See Fig.~\ref{fig:dh-statistics}). 

Using the method introduced in the previous section, by comparing Eq.\eqref{eq:Pgood-dh} of the ideal scheme and the $Q$-function of coherent state,
\begin{equation}
    |\langle\beta|\alpha\rangle|^2=\exp\left[-
    (\beta_1-\alpha_1)^2-(\beta_2-\alpha_2)^2
    \right], %ez 2 show
\end{equation}
where for complex numbers we introduced the notation $\alpha=\alpha_1+i\alpha_2$, and introducing 
\begin{align}
\begin{split}
    \sigma_{D_1}\equiv\frac{1}{D_1}-1,\\
    \sigma_{D_2}\equiv\frac{1}{D_2}-1,
\end{split}
\label{eq:dh-G}
\end{align}
we may write $P_G$ in the form
\begin{multline}
    {{P}}_G=\frac{1}{4{\pi}C_3S_3[\eta_1+\eta_4][\eta_3+\eta_2]S_2^2\sqrt{ C_1C_2S_4C_4}|\alpha_L|^2} \int \mathop{dx'_1dx'_2}\times\\\times G(x_1-x_1', \sigma_{D_1})G(x_2-x_2', \sigma_{D_2})|\langle\alpha|e^{-i\phi_L}\left(-x_1'+ix_2'\right)\rangle|^2,%\label{eq:dh-povm}
\end{multline}
where $|e^{-i\phi_L}\left(-x_1'+ix_2'\right)\rangle$ is a coherent state. This allows us to write the respective POVM as
\begin{multline}
    \hat{{P}}_G=\frac{1}{4{\pi}C_3S_3[\eta_1+\eta_4][\eta_3+\eta_2]S_2^2\sqrt{C_1C_2S_4C_4}|\alpha_L|^2} \int \mathop{dx'_1dx'_2}\times\\\times G(x_1-x_1', D_1)G(x_2-x_2', D_2)|e^{-i\phi_L}\left(-x_1'+ix_2'\right)\rangle\langle e^{-i\phi_L}\left(-x_1'+ix_2'\right)|.\label{eq:dh-povm}
\end{multline}
Note the difference between definitions of variances of Gaussian functions Eq.~\eqref{eq:h-G} and Eq.~\eqref{eq:dh-G}, namely the difference in variances by a factor of $2$. This is due to $Q$-functions with which we compare approximations with ideal parameters having (and not having) an explicit factor of $2$ in their variances.

We may also see that $D_1$ and $D_2$, defined by Eq.~\eqref{eq:dh-D}, may be greater than $1$ for some parameters of the system. This becomes apparent once we present $D_1$ ($D_2$) as $2C_1^2D$ ($2S_3^2D$), with $D$ defined as per Eq.~\eqref{eq:defs-x-D}. This is an issue due to the variances $\sigma_{D_1}$ and $\sigma_{D_2}$ given by Eq.~\eqref{eq:dh-G}, being negative for these parameters, meaning that any asymmetry in BS$_1$ and BS$_3$ is not described by POVM given by Eq.~\eqref{eq:dh-povm}. \hl{Evidently, using coherent states for this case is flawed, see Ref.{~\cite{doi:10.1080/09500348714550131}}, where, for similar case of non-balanced BS$_{1,3}$, POVM is derived to be a projector onto squeezed coherent state.} This problem requires further investigation and is beyond the scope of this work.

\begin{figure}
    \centering
    \begin{minipage}[c]{.75\linewidth}
\includegraphics[width=\linewidth]{pics/double-homodyne/full.pdf}
\subcaption[]{}
        \end{minipage}
\hfill
        \begin{minipage}[c]{.45\linewidth}
 \includegraphics[width=\linewidth]{pics/double-homodyne/dm1.pdf}
\subcaption[]{}
\end{minipage}
\hfill
        \begin{minipage}[c]{.45\linewidth}
 \includegraphics[width=\linewidth]{pics/double-homodyne/dm2.pdf}
\subcaption[]{}
\end{minipage}
    \caption{(a) -- 
     Numerically calculated exact statistical distribution of photon count difference in asymmetrical double homodyne scheme, given by  Eq.~\eqref{eq:accurate-dh}, and calculated distance between this distribution and approximate distribution, given by Eq.~\eqref{eq:Pgood-dh}; (b), (c) -- Projections of exact (blue) and approximate (black) statistical distributions of photon count difference, all calculated for parameters $\alpha=0.25+0.25i$, $\alpha_L=5$, $\eta_1=\eta_3=1$ and $\eta_2=\eta_4=0.75$.
    }\label{fig:dh-statistics}
\end{figure}
%\input{0Experiment}

%%%%%%%%%%%%%%%%%%%%%%%%%%%%
\section{Double homodyne detection}
\label{sec:double-homodyne}
%%%%%%%%%%%%%%%%%%%%%%%%%%%%%%

In the section,
we consider the eight-port double homodyne scheme
depicted in Fig.~\ref{fig:double-homodyne}
(see, e.g., Refs.~\cite{Vogel:bk:2006,lahti2010realistic}).
This measurement scheme is known to allow reconstructing the
Husimi $Q$ function of the signal state~\cite{Richter:98} providing complete information about the signal
state that may be used in CV-QKD protocols. It should be noted that restoration of the
complex amplitude of the state completely serves as the basis for composable security
proofs~\cite{PhysRevLett.93.170504,PhysRevLett.118.200501}.
%~\cite{pirandola2024improvedcomposablekeyrates,pascualgarcía2024improvedfinitesizekeyrates}.


Figure~\ref{fig:double-homodyne} shows two homodyne setups
with elements such as beam splitters BS$_i$
and photodetectors $D_{1,2}^{(i)}$ labeled by the index $i\in\{1,2\}$.
Referring to Fig.~\ref{fig:double-homodyne},
the LO and signal modes are transmitted through
the beam splitters BS$_L$ and BS$_S$, respectively.
Similar to the analysis performed in the previous section,
we assume that the modes are in the coherent states,
$\ket{\alpha}$ and $\ket{\alpha_L}$. So, we
have the amplitudes 
\begin{align}
  &
    \label{eq:amplitudes-SL-i}
    \alpha^{(1)}=C_S\alpha,\quad \alpha^{(2)}=S_S\alpha,
    \notag
  \\
  &
    \alpha_L^{(1)}=C_L\alpha_L,\quad \alpha_L^{(2)}=-iS_L\alpha_L,
\end{align}
where $\alpha^{(i)}$ ($\alpha_{L}^{(i)}$) stands for the amplitude
describing the input coherent state of the signal (LO) mode
of the homodyne labeled by the upper index $i\in \{1,2\}$. 
Note that the phase factor $-i=e^{-i\pi/2}$ in the expression for
$\alpha^{(2)}_L$ is introduced by
a suitably chosen phase shifter placed before
the corresponding input port
of the beam splitter BS$_2$.

Direct calculation shows that
the joint statistics of the difference photocount events
is determined by the product
of the Skellam distributions given by
\begin{align}
  &
    \label{eq:P_mu1m2}
    \prob(\mu_1,\mu_2)=\prob_1(\mu_1)\prob_2(\mu_2),
    \quad \mu_i=m_1^{(i)}-m_2^{(i)},
  \\
  &
  \label{eq:Skel-i}
  \prob_i(\mu_i)=e^{-\eta_1^{(i)}|\alpha_1^{(i)}|^2}e^{-\eta_2^{(i)}|\alpha_2^{(i)}|^2}
    \Biggl(\frac{\eta_1^{(i)}|\alpha_1^{(i)}|^2}{\eta_2^{(i)}|\alpha_2^{(i)}|^2}\Biggr)^{\mu_i/2}
    \notag
  \\
  &
    \times
I_{\mu_i}\bigl(2\sqrt{\eta_1^{(i)}\eta_2^{(i)}|\alpha_1^{(i)}|^2|\alpha_2^{(i)}|^2}\bigr),
\end{align}
where, similar to the homodyne scheme,
the amplitudes of the coherent states at the output ports of the beam splitter
BS$_i$
\begin{align}
  &
    \label{eq:amplitudes-i}
    \alpha_1^{(i)}=C_i\alpha^{(i)}+S_i\alpha_L^{(i)},\quad
    \alpha_2^{(i)}=-S_i\alpha^{(i)}+C_i\alpha_L^{(i)}
\end{align}
are expressed in terms of the transmission and reflection amplitudes,
$C_i=\cos\theta_i$ and $S_i=\sin\theta_i$.

We can now apply Eqs.~\eqref{eq:Gaussian}-\eqref{eq:mu_G}
to approximate each Skellam distribution on
the right hand side of Eq.~\eqref{eq:P_mu1m2}
and derive the Gaussian approximation
for the double homodyne scheme in the form:
\begin{align}
  &
  \label{eq:Gaussian-8p}
    \prob_G(\mu_1,\mu_2)=G(\mu_1-\mu_G^{(1)};\sigma_G^{(1)})G(\mu_2-\mu_G^{(2)};\sigma_G^{(2)}),
    \\
  &
    \label{eq:sigma_G-i}
    \sigma_G^{(i)}=
    (\eta_1^{(i)}S_i^2+\eta_2^{(i)}C_i^2) |\alpha_L^{(i)}|^2,
  \\
  &
    \label{eq:mu_G-i}
    \mu_G^{(i)}=
    (\eta_1^{(i)}S_i^2-\eta_2^{(i)}C_i^2) |\alpha_L^{(i)}|^2
    +C_iS_i (\eta_1^{(i)}+\eta_2^{(i)})|\alpha_L^{(i)}| 
    \notag
  \\
  &
    \times
    2\Re\alpha^{(i)} e^{-i\phi},\quad \phi=\arg\alpha_L.
\end{align}
Similar to Eq.~\eqref{eq:Pgood-w-def},
it is useful to put
the probability~\eqref{eq:Gaussian-8p}
into the folowing quadrature form:
\begin{align}
  \label{eq:P_G-x1x2}
  &
    \prob_G(x_1,x_2)=\frac{1}{2\pi\sqrt{\sigma_G^{(1)}\sigma_G^{(2)}}}
    \exp\Bigl\{
    -\frac{(x_1-\Re\alpha e^{-i\phi})^2}{\sigma_1}
    \notag
  \\
  &
    -\frac{(x_2-\Im\alpha e^{-i\phi})^2}{\sigma_2}
    \Bigr\}
\end{align}
where $x_i$ are the quadrature variables given by
\begin{align}
  &
  \label{eq:x-1}
  x_1=
  \frac{1}{2(\eta_1^{(1)}+\eta_2^{(1)})C_1S_1C_S}
  \notag
  \\
  &
    \times
  \Bigl\{
  \frac{\mu_1}{|\alpha_L^{(1)}|}-(\eta_1^{(1)}S_1^2-\eta_2^{(1)}C_1^2) |\alpha_L^{(1)}|
    \Bigr\},
  \\
  &
  \label{eq:x-2}
  x_2=
  \frac{1}{2(\eta_1^{(2)}+\eta_2^{(2)})C_2S_2S_S}
  \notag
  \\
  &
    \times
  \Bigl\{
  \frac{\mu_2}{|\alpha_L^{(2)}|}-(\eta_1^{(2)}S_2^2-\eta_2^{(2)}C_2^2) |\alpha_L^{(2)}|
    \Bigr\},
\end{align}
and relations
\begin{align}
  &
  \label{eq:sigm-i}
    \sigma_1=\frac{\sigma_x^{(1)}}{ 2 C_S^2},
    \quad
    \sigma_2=\frac{\sigma_x^{(2)}}{ 2 S_S^2},
  \\
  &
    \label{eq:sigm-x-i}
    \sigma_x^{(i)}=\frac{\eta_1^{(i)}S_i^2+\eta_2^{(i)}C_i^2}{
    \left[(\eta_1^{(i)}+\eta_2^{(i)})C_iS_i\right]^2
    }
\end{align}
give the quadrature variances $\sigma_1$ and $\sigma_2$.

\begin{figure*}
    \centering
    \begin{subfigure}[c]{.3\linewidth}
\includegraphics[width=\linewidth]{pics/double-homodyne/1111.pdf}
\caption[]{$\eta_1^{(i)}=1,\eta_2^{(i)}=1$}
        \end{subfigure}
\hfill
        \begin{subfigure}[c]{.3\linewidth}
 \includegraphics[width=\linewidth]{pics/double-homodyne/10.510.5.pdf}
\caption[]{$\eta_1^{(i)}=1,\eta_2^{(i)}=0.5$}
\end{subfigure}
\hfill
    \begin{subfigure}[c]{.3\linewidth}
\includegraphics[width=\linewidth]{pics/double-homodyne/1110.5.pdf}
\caption[]{$\eta_1^{(i)}=1,\eta_2^{(1)}=1,\eta_2^{(2)}=0.5$}
        \end{subfigure}
        \caption{Double homodyne statistical distribution of photocount differences
          computed from Eq.~\eqref{eq:P_mu1m2}
          for various detector efficiencies at $\alpha=0.5$ and $\alpha_L=5$.
          All the beam splitters are taken to be balanced.
}
\label{fig:dh-statistics}
\end{figure*}


As in Sec.~\ref{sec:homodyne},
formula~\eqref{eq:P_G-x1x2}
giving the $Q$-symbol of POVM (see Eq.~\eqref{eq:PG-as-avr})
provides the starting point for reconstruction of the POVM
describing the double homodyne measurements.
In the ideal case, where all the beam splitters are balanced
and the photodetection is perfect,
we have
\begin{align}
  &
  \label{eq:P0-hetero}
    \prob_{G}^{(0)}(x_1,x_2)=\frac{|\avr{z|\alpha}|^2}{\pi |\alpha_L|^2},\quad
|\avr{z|\alpha}|^2=e^{-|z-\alpha|^2},
\end{align}
where
\begin{align}
  \label{eq:z}
    z=(x_1+ix_2) e^{i\phi},\quad     x_i=\frac{\mu_i}{|\alpha_L|}.
\end{align}
So, the POVM is proportional to a projector onto the coherent state
$\ket{z}\equiv\ket{(x_1+ix_2)e^{i\phi}}$:
\begin{align}
  \label{eq:POVM-hetero-ideal}
    \hat{\Pi}_G^{(0)}(x_1,x_2)=\frac{1}{\pi |\alpha_L|^2} \ket{z}\bra{z}.
\end{align}
For non-ideal measurements, the probability~\eqref{eq:P_G-x1x2}
can be expressed
as a Gaussian superposition of the coherent state Husimi functions
with the help of the convolution relation~\eqref{eq:convolution}
as follows
\begin{align}
  \label{eq:P_G-x1x2-2}
  &
    \prob_G(x_1,x_2)=\frac{\sqrt{\sigma_1\sigma_2}}{2\pi\sqrt{\sigma_G^{(1)}\sigma_G^{(2)}}}
\int \dd \beta_1\dd \beta_2
    G(x_1-\beta_1;\sigma_N^{(1)})
    \notag
  \\
  &
    \times
    G(x_2-\beta_2;\sigma_N^{(2)})
    |\avr{\beta e^{i\phi}|\alpha}|^2,\quad \beta=\beta_1+i\beta_2,
\end{align}
where the excess noise variance
$\sigma_N^{(i)}$ is determined by the relation
\begin{align}
  &
    \label{eq:sigm-N-i}
    2\sigma_{N}^{(i)}=\sigma_i-1.
\end{align}
The corresponding expression for the POVM reads
\begin{align}
  &
  \label{eq:povm-hetero}
    \hat{\Pi}_G(x_1,x_2)=\frac{\sqrt{\sigma_1\sigma_2}}{2\pi\sqrt{\sigma_G^{(1)}\sigma_G^{(2)}}}
\int \dd \beta_1\dd \beta_2
    G(x_1-\beta_1;\sigma_N^{(1)})
    \notag
  \\
  &
    \times
    G(x_2-\beta_2;\sigma_N^{(2)})
    \ket{\beta e^{i\phi}}\bra{\beta e^{i\phi}}.
\end{align}

An important point is that
the results given by Eq.~\eqref{eq:P_G-x1x2-2}
and Eq.~\eqref{eq:povm-hetero}
are well defined only if $\sigma_1$ and $\sigma_2$
are both above unity,
so that the excess noise variances~\eqref{eq:sigm-N-i} are
positive.
From Eq.~\eqref{eq:sigm-i},
this requires that conditions
$\sigma_x^{(1)}\ge 2 C_S^2$
and $\sigma_x^{(2)}\ge 2 S_S^2$ be met.

When the beam splitter BS$_S$
is balanced $2 C_S^2=2 S_S^2=1$
and inequality (see Eq.~\eqref{eq:sigma-x-min})
\begin{align}
  \label{eq:sigma-x-min-i}
  \sigma_x^{(i)}\ge \Biggl(
\frac{\sqrt{\eta_1^{(i)}}+\sqrt{\eta_2^{(i)}}}{\eta_1^{(i)}+\eta_2^{(i)}}
  \Biggr)^2\ge 1
\end{align}
ensures applicability of the expression for the POVM.
Otherwise, either $2 C_S^2$ or $2 S_S^2$ will be above unity,
and our results are valid only if
the value of the corresponding variance $\sigma_x^{(i)}$ is sufficiently high.
For example, at $\eta_{1,2}^{(i)}=\eta<1/2$,
the minimal values of $\sigma_x^{(i)}$ are higher than $2$
(see Eq.~\eqref{eq:sigma-x-min-i})
and the noise variance will be positive at
any disbalance of the signal mode beam splitter
because $\max\{2 C_S^2,2 S_S^2\}\le 2$.
When the noise variance is negative,
the POVM reconstruction procedure
needs to be generalized.
We shall present details on this generalization
in the next section.
Meanwhile,
in the remaining part of this section,
we confine ourselves to the cases where $\sigma_N^{(i)}$ are positive.

The effects of photodetection asymmetry
are illustrated in Fig.~\ref{fig:dh-statistics}
which presents numerical results for
the double homodyne distribution~\eqref{eq:P_mu1m2}
in the photocont difference $\mu_1$-$\mu_2$ plane.
Referring to Fig.~\ref{fig:dh-statistics},
in addition to the shift of the distribution,
the asymmetry induced difference of the variances
at $\eta_1^{(1)}+\eta_2^{(1)}\ne \eta_1^{(2)}+\eta_1^{(2)}$
manifests itself as the two dimensional anisotropy
of the double homodyne distribution.

%%%%%%%%%%%%%%%%%%%%%%%%%%%%
\section{Positive operator-valued measure and squeezed states}
\label{sec:gen-POVM}
%%%%%%%%%%%%%%%%%%%%%%%%%%%%%%

From Eq.~\eqref{eq:sigm-N-i},
the expression for the POVM
in the form of incoherent gaussian superposition
of coherent states is justified only if
both the quadrature variances, $\sigma_1$ and $\sigma_2$,
exceed unity.
In this section, we show that our procedure employed
for derivation of the double homodyne POVM can be
suitably generalized by enlarging a set of the pure states to include
the squeezed coherent states
\begin{align}
  \label{eq:squeezed-def}
  \ket{\beta,\zeta}=\hat{D}(\beta)\hat{S}(\zeta)\ket{0},
\end{align}
where $\hat{D}(\beta)$ ($\hat{S}(\zeta)$) is the displacement
(squeezing) operator  given by
\begin{align}
\label{eq:D-S-def}
\hat{D}(\beta) =
e^{\beta\hcnj{\hat{a}}-\beta^*\hat{a}}, \quad
    \hat{S}(\zeta)=e^{\frac{1}{2}\left(\zeta\hat{a}^{\dag2}-\zeta^*\hat{a}^2\right)},
\end{align}
$\beta$ and $\zeta$ are the complex-valued amplitude
and the squeeze parameter, respectively. 

To this end, we consider the case, where
the squeeze parameter is given by
\begin{align}
  \label{eq:sq-param}
  \zeta=r e^{2 i \phi},\quad
  r\in\mathbb{R}
\end{align}
and the non-normalized Husimi distribution
for the squeezed state~\eqref{eq:squeezed-def} takes
the form
(see, e.g., the textbook~\cite{Gerry:bk:2005})
\begin{align}
\label{eq:squeezed-Q}
  &
    |\langle\beta,r e^{2 i \phi}|\alpha\rangle|^2=\frac{1}{\cosh r}
    \exp\Bigl\{-
    \frac{e^{-r}}{\cosh r}{}
    \left(\tilde{\beta}_1e^{r}-\tilde{\alpha}_1\right)^2
    \notag
  \\
    &-\frac{e^{r}}{\cosh r}{}\left(\tilde{\beta}_2e^{{-r}}-\tilde{\alpha}_2\right)^2
      \Bigr\},
  \\
  &
    \label{eq:tbeta-alpha}
    \tilde{\beta}=\tilde{\beta}_1+ i \tilde{\beta}_2=\beta e^{-i\phi},
    \quad
    \tilde{\alpha}=\tilde{\alpha}_1+ i \tilde{\alpha}_2=\alpha e^{-i\phi}.
\end{align}
By using
this squeezed state distribution instead
of the coherent state one 
given in Eq.~\eqref{eq:POVM-hetero-ideal},
we are led to the expressions for
the noise variances modified as follows
\begin{align}
  \label{eq:sigma_N-sq-1}
  &
    2\sigma_N^{(1,2)}=\sigma_{1,2}-e^{\pm r}\cosh r.
\end{align}
These expressions present the
extension of the relations~\eqref{eq:sigm-N-i}
to the case with non-vanishing squeeze parameter.
As an immediate consequence of Eq.~\eqref{eq:sigma_N-sq-1},
we find that
the conditions for the noise variances to be positive definite
can be written in the form of two inequalities 
\begin{align}
  \label{eq:sigma_N-sq-2}
  &
    4\sigma_N^{(1)}=\delta_1-e^{2r}\ge 0,
    \quad
    4\sigma_N^{(2)}=\delta_2-e^{-2r}\ge 0,
\end{align}
where the quadrature variance parameters
\begin{subequations}
  \label{eq:delta_12}
\begin{align}
  \label{eq:delta_1}
  &
    \delta_1=2\sigma_1-1=(q+1)\sigma_x^{(1)}-1\ge q=\frac{S_S^2}{C_S^2},
  \\
  &
    \label{eq:delta_2}
    \delta_2=2\sigma_2-1=(q^{-1}+1)\sigma_x^{(2)}-1\ge q^{-1}=\frac{C_S^2}{S_S^2}
\end{align}
\end{subequations}
are expressed in terms of
the disbalance
(reflection-to-transmission)
ratio of the input beam splitter, $q$,
and the parameters $\sigma_{x}^{(1)}$
and $\sigma_{x}^{(2)}$
(see Eq.~\eqref{eq:sigm-x-i})
that cannot be smaller than unity
(see Eq.~\eqref{eq:sigma-x-min-i}): $\sigma_x^{(i)}\ge 1$.

In our subsequent analysis,
we assume without the loss of generality
that the reflectance of the input beam splitter BS$_S$ is
larger than its transmittance,
so that the disbalance ratio is above unity 
\begin{align}
  \label{eq:q-param}
  &
    q=\frac{1-C_S^2}{C_S^2}=\frac{S_S^2}{C_S^2}\ge 1.
\end{align}
As is shown in Fig.~\ref{fig:squeezing},
this implies that the variance $\delta_1$
is above unity: $\delta_1\ge q\ge 1$,
whereas the minimal value of $\delta_2$
is $q^{-1}\le 1$.
It is illustrated that
the noise variances are positive
provided the squeeze parameter $r$
is ranged between
the endpoints of the interval given by
\begin{align}
  \label{eq:r1-r2}
  &
    r\in[r_2,r_1],\;
    r_1=\ln\delta_1^{1/2},
    \;
    r_2=\ln\delta_2^{-1/2}%\max\{\ln\delta_2^{-1/2},0\}.
\end{align}

\begin{figure}
    \centering
    \includegraphics[width=0.9\linewidth, page={2}]{tikz_article.pdf}
    \caption{An illustration for the conditions for the noise variances, see Eqs.~\eqref{eq:sigma_N-sq-2}-\eqref{eq:r1-r2}.
    Solid grey lines represent the exponents $e^{\pm2r}$. Dashdotted black, red and blue lines are the independent on $r$ functions $q$ and $q^{-1}$, $\delta_1$ and $\delta_2$ respectively. Solid purple line is the interval of squeezing parameter $r\in[r_2,r_1]$, solid red (blue) line %with pointing arrow of the same color 
    is the magnitude of $4\sigma_N^{(1)}$ ($4\sigma_N^{(2)}$) at $r$.}
    \label{fig:squeezing}
\end{figure}

An important point is that
the value of the squeeze parameter
is not uniquely determined
by the conditions~\eqref{eq:sigma_N-sq-2}.
We have a unique value of
the squeeze parameter only in  the limiting case
of perfect homodyne measurements
with $\sigma_{x}^{(1)}=\sigma_{x}^{(2)}=1$
and $\delta_{1}=\delta_{2}^{-1}=q$.
In this case, the squeeze parameter is
unambiguously defined
\begin{align}
  &
  \label{eq:r-ideal}
  r_1=r_2=\rho=\frac{1}{2}\ln q=\ln\sqrt{\frac{1-C_S^2}{C_S^2}}
\end{align}
and the probability~\eqref{eq:P_G-x1x2}
expressed in terms of the distribution~\eqref{eq:squeezed-Q}
\begin{align}
  &
    \label{eq:P_G-ideal}
  \prob_{G}^{(1)}(x_1,x_2)=\frac{\cosh \rho}{2\pi |\alpha_L^{(1)}\alpha_L^{(2)}|}
    |\avr{\beta e^{i\phi},\rho e^{2i\phi}|\alpha}|^2,
  \\
  &
    \label{eq:beta-ideal}
    \beta=e^{-\rho}x_1+i e^{\rho} x_2
\end{align}
yields the POVM
\begin{align}
    \label{eq:Pi_G-ideal}
    \hat{\Pi}_{G}^{(1)}(x_1,x_2)&=\frac{\cosh \rho}{2\pi |\alpha_L^{(1)}\alpha_L^{(2)}|}
    \notag
  \\
  &
  \times
    \ket{\beta e^{i\phi},\rho e^{2i\phi}}\bra{\beta e^{i\phi},\rho e^{2i\phi}},
\end{align}
which is propotional to the pure squeezed state
$\ket{\beta e^{i\phi},\rho e^{2i\phi}}$.
The result~\eqref{eq:r-ideal}
was reported in Ref.~\cite{genoni2014general}.

When the homodyne measurements are not
perfect due to unbalanced beam splitters and nonideal photodetectors,
the squeeze parameter is no longer uniquely defined.
So, in the interval~\eqref{eq:r1-r2},
we have the decomposition of
the probability~\eqref{eq:P_G-x1x2}
\begin{align}
  \label{eq:P_G-gen-x1x2-2}
  &
    \prob_G(x_1,x_2)=\frac{\sqrt{\sigma_1\sigma_2}}{2\pi\sqrt{\sigma_G^{(1)}\sigma_G^{(2)}}}
\notag
  \\
  &
    \times
    \int \dd \beta_1\dd \beta_2
    G(x_1e^{-r}-\beta_1;\sigma_N^{(1)}(r)e^{-2r})
    \notag
  \\
  &
    \times
    G(x_2e^{r}-\beta_2;\sigma_N^{(2)}(r)e^{2r})
    |\avr{\beta e^{i\phi},r e^{2i\phi}|\alpha}|^2 
\end{align}
that varies with the squeeze parameter $r$.
Similarly,
the corresponding POVM 
\begin{align}
  \label{eq:Pi_G-gen-x1x2}
  &
    \hat{\Pi}_G(x_1,x_2)=\frac{\sqrt{\sigma_1\sigma_2}}{2\pi\sqrt{\sigma_G^{(1)}\sigma_G^{(2)}}}
\notag
  \\
  &
    \times
    \int \dd \beta_1\dd \beta_2
    G(x_1e^{-r}-\beta_1;\sigma_N^{(1)}(r)e^{-2r})
    \notag
  \\
  &
    \times
    G(x_2e^{r}-\beta_2;\sigma_N^{(2)}(r)e^{2r})
    \notag
  \\
  &
    \times
    \ket{\beta e^{i\phi},r e^{2i\phi}}\bra{\beta e^{i\phi},r e^{2i\phi}}
\end{align}
decomposed into
the Gaussian incoherent superposition
of the pure squeezed states explicitly depends on the value of $r$.

Constructing POVM~\eqref{eq:Pi_G-gen-x1x2} covariance matrix from Eq.~\eqref{eq:P_G-gen-x1x2-2} yields
\begin{align}
  \label{eq:Pi_dh}
  &\Sigma^{\ind{DH}}=
    \begin{pmatrix}
      e^{2r}&0\\0&e^{-2r}
    \end{pmatrix}+
    \begin{pmatrix}
        \delta_1-e^{2r}&0\\
        0&\delta_2-e^{-2r}
    \end{pmatrix}\notag\\ &\equiv \Sigma_0^{\ind{DH}}+\Sigma_N^{\ind{DH}},
\end{align}
where, similarly to Eq.~\eqref{eq:Pi_h}, we defined matrices of perfect measurement $\Sigma_0^{\ind{DH}}$ and excess noise $\Sigma_N^{\ind{DH}}$. Note that the condition~\eqref{eq:sigma_N-sq-2} is equivalent to the requirement of $\Sigma_N^{\ind{DH}}$ being positive.

Our analysis suggests that  the ambiguity (non-uniqueness) of
the Gaussian represention~\eqref{eq:Pi_G-gen-x1x2} for the double-homodyne
POVM is a universal feature coming into play
in the presence of imperfections.
Even when $\delta_2\ge 1$ and
the coherent-state represention~\eqref{eq:povm-hetero} for the POVM is well-defined,
the photocount statistics can be reproduced using
the squeezed-state representation~\eqref{eq:Pi_G-gen-x1x2}
with $0<r \le r_1$.
One of the way to make the Gaussian POVM decomposition unique
is to place additional constraints
on the noise variances that would fix the value of the squeeze parameter.
For example, the ``amount'' of noise, i.e. the product of variances $\sigma_N^{(i)}$, is maximized by the midpoint of the interval $\frac{r_1+r_2}2$.
%the squeeze parameter would take the minimal (maximal) value: $r=\min\{r_1,r_2\}$ ($r=\max\{r_1,r_2\}$)
%provided the constraint requires minimization of
%the noise variance $\sigma_N^{(2)}$ ($\sigma_N^{(1)}$).
%%%%%%%%%%%%%%%%%%%%%%%%%%%%%
\section{Incorporating measurement imperfections into the GG02 CV-QKD Protocol}
\label{sec:protocol}

In this section, we apply the developed formalism for asymmetric measurements to analyze the Gaussian-modulated coherent-state (GMCS, or GG02~\cite{PhysRevLett.88.057902}) CV-QKD protocol. We evaluate the mutual information, Holevo information, and asymptotic secret fraction (secure key rate per symbol) under the untrusted-noise scenario to quantify how measurement asymmetry affects protocol security.

In the GG02 protocol, the mutual information between Alice and Bob, when Bob performs homodyne detection, is given by
\begin{align}
  \label{eq:I_H_res}
  I_{\ind{AB}}^{\ind{H}}=\frac{1}{2}\log\left[1+\frac{4TV_{\ind{A}}}{\sigma_x+2\xi}\right],
\end{align}
where $T$ is the channel transmission, $\xi$ is the channel noise variance, $V_{\ind{A}}$ is Alice's modulation variance, and $\sigma_x$ is given by Eq.~\eqref{eq:sigma-x}.
If Bob uses double-homodyne detection, mutual information takes the form
\begin{align}
  \label{eq:I_DH_res}
  I_{\ind{AB}}^{\ind{DH}}=\frac{1}{2}\sum_i\log\left[1+\frac{2TV_{\ind{A}}}
    {\sigma_i+\xi}\right],
\end{align}
where $\sigma_i$ are given by Eq.~\eqref{eq:sigm-x-i}.
The details of these derivations are given in Appendix~\ref{sec:MI-and-EBtoPm}.

Next, we compute the Holevo information in the untrusted-noise scenario, in which detection imperfections are modeled as additional noise accessible to the eavesdropper (Eve). We assume that Eve controls a Gaussian channel that modifies only the covariance matrix, leaving the mean unchanged, while Bob’s measurements are considered ideal. %This additional noise effectively enhances the quantum channel parameters, thereby increasing the information accessible to Eve. %Consequently, this noise contribution effectively augments the quantum-channel parameters, increasing Eve’s accessible information.
Using the equivalence between the prepare-and-measure (PM) and entanglement-based (EB) pictures (see App.~\ref{sec:MI-and-EBtoPm}), the Holevo information can be derived from the covariance matrix of the two mode squeezed vacuum (TMSV) state $\Sigma^{\ind{EB}}_{\ind{AB}}$. 
%It is given given by
%\begin{align}
%  \label{eq:TVMS}
%    \Sigma^{\ind{TMSVS}}=\begin{pmatrix}
%        V\mathbb{I}&\sqrt{V^2-1}\sigma_Z\\
%        \sqrt{V^2-1}\sigma_Z&V\mathbb{I}
%    \end{pmatrix},
%\end{align}
After transmitting through noisy channel
(see Ref.~\cite{laudenbach2018continuous}), TMSV covariance matrix reads
\begin{align}
  \label{eq:TVMS-final}
    &\Sigma^{\ind{EB}}_{\ind{AB}}=\begin{pmatrix}
        V\mathbb{I}&\sqrt{T\left(V^2-1\right)}\sigma_Z\\
        \sqrt{T\left(V^2-1\right)}\sigma_Z&\left[T(V-1)+1+2\xi\right]\mathbb{I}
    \end{pmatrix}
    \notag
    \\&\equiv\begin{pmatrix}
        V\mathbb{I}&c\sigma_Z\\
        c\sigma_Z&V_{\ind{B}}\mathbb{I}
    \end{pmatrix},
\end{align}
where $\sigma_Z=\ind{diag}(1,-1)$, and
\begin{align}
  \label{eq:V(V_A)}
    V=1+4V_{\ind{A}}.
\end{align}
%Note that Bob's variance $V_{\ind{B}}$ is exactly the same as Bob's variance calculated in PM protocol per Eq.~\eqref{eq:sigma-hom-def} after substituting $\sigma_x=1$, as is required to equate both protocols.
%Measurement noise modifies this covariance matrix as
In the paradigm described previosly, Eve's entropy, $S_{\ind{E}}=S_{\ind{AB}}$ will be calculated from symplectic eigenvalues of the following covariance matrix:
\begin{align}
    \label{eq:sigma_m_cov}
\Sigma_{\ind{AB}}^{(m)}=\begin{pmatrix}
    V\mathbb{I}&c\sigma_z\\
    c\sigma_z&V_{\ind{B}}\mathbb{I}+\Sigma_N^{\ind{(m)}}
\end{pmatrix},\quad m\in\{\ind{H},\ind{DH}\}
\end{align}
where $\Sigma^{\ind{H}}_N$ ($\Sigma^{\ind{DH}}_N$) is given by Eq.~\eqref{eq:Pi_h} (Eq.~\eqref{eq:Pi_dh}).
%from symplectic eigenvalues of which we can calculate Eve's entropy, $S_{\ind{E}}=S_{\ind{AB}}$.

The conditional entropy $S_{\ind{E|B}}=S_{\ind{A|B}}$ is found via partial measurement formula~\cite{Serafini:bk:2017}, written for $\Sigma_{\ind{AB}}^{(m)}$ as follows:
\begin{align}
  \label{eq:serf-a|b}
        \Sigma_{\ind{A|B}}^{(m)}=
        V\mathbb{I}-c^2\sigma_z(V_{\ind{B}}\mathbb{I}+\Sigma^{(m)})^{-1}\sigma_z,\quad m\in\left\{\ind{H}, \ind{DH}\right\}.
    \end{align}
For homodyne detection, calculating symplectic eigenvalue of conditional covariance matrix as square root of the determinant  yields:
\begin{align}
     \label{eq:nu3}
     &\nu_3^{\ind{H}}=\sqrt{V\left(V-\frac{c^2}{V_{\ind{B}}+\sigma_N}\right)}.
 \end{align}
The same calculation for double homodyne results in:
\begin{align}
    \label{eq:nu3-dh}
    & \nu_3^{\ind{DH}}=V\sqrt{\frac{\left(V_{\ind{B}}  + \delta_1 - \frac{c^{2}}V\right) \left( V_{\ind{B}}  + \delta_2 - \frac{c^{2}}V\right)}{\left( V_{\ind{B}} + \delta_1\right) \left(V_{\ind{B}} + \delta_2\right)}}.
\end{align}

The Holevo information is thus
\begin{align}
\label{eq:chi}
    &\chi_{\ind{EB}}\equiv S_{\ind{E}}-S_{\ind{E}|\ind{B}}=S_{\ind{AB}}-S_{\ind{A}|\ind{B}}\notag\\&=\sum_{i=1,2}g(\nu_i)-g(\nu_3),
\end{align}
where
\begin{align}
\label{eq:g(nu)}
    g(\nu)=\frac{\nu+1}{2}\log \frac{\nu+1}{2}-
    \frac{\nu-1}{2}\log\frac{\nu-1}{2},
\end{align}
with appropriate substitution of $\nu_i$ depending on the measurement scheme.

Note that in the case of double homodyne detection, joint entropy $S_{\ind{AB}}$ and thereby Holevo information $\chi_{\ind{EB}}$ are explicitly dependent on squeezing parameter $r\in[r_2,r_1]$~\eqref{eq:r1-r2}, as seen from the expression for covariance matrix~\eqref{eq:sigma_m_cov}. Moreover, as illustated by Fig.~\ref{fig:Entropy}, the choice of $r$ significantly affects the value of $\chi_{\ind{EB}}$.
Regarding Fig.~\ref{fig:Entropy}, for ideal double homodyne only $r=0$ is allowed (black dot), yielding respective eigenvalues. When detection efficiencies are unity and $C_S^2 \neq 0.5$, the singular allowed value of $r$ shifts to $r=\rho \neq 0$~\eqref{eq:r-ideal}, as shown at point $\rho$ in Fig.~\ref{fig:squeezing}. In this case, the value of $S_{\ind{AB}}$ is independent of $\rho$ and determined solely by modulation variance $V_{\ind{A}}$ and channel parameters $T$ and $\xi$.
In the symmetrical but non-ideal case (red curve), symmetrical interval $[-r_1,r_1]$ arises, and joint entropy is maximized by the midpoint $r=0$. In this case, choosing the POVM in the form of coherent state superposition~\eqref{eq:povm-hetero} is not only allowed, but optimal.
In asymmetrical case, if $C_S^2=0.5$ (blue curve), while using $r=0$, i.e. Eq.~\eqref{eq:povm-hetero}, is possible, joint entropy is maximized by $r\neq0$. Finally, in asymmetrical case with $C_S^2\neq0.5$ (green curve), usage of $r=0$ results in one of eigenvalues being less than $1$, which is impossible. Therefore, the more general representation given by Eq.~\eqref{eq:Pi_G-gen-x1x2} must be employed.
For calculations, we will optimize $r$ to maximize $\chi_{\ind{EB}}$.
\begin{figure}
  \centering
  \includegraphics[width=.75\linewidth, ]{pics/S_AB.pdf}
  \caption{Joint entropy $S_{\ind{AB}}$ as a function of squeezing parameter $r$ for various detector efficiencies. Unless specified, all beam spliiters are taken to be balanced. Parameters: $V_{\ind{A}}=1
        , T=0.95, \xi=0.5\cdot10^{-3}$.}\label{fig:Entropy}
\end{figure}

The asymptotic secret fraction is
\begin{equation}
\label{eq:r}
    K=\beta I_{AB}-\chi_{EB},
\end{equation}
where $\beta$ is the reconciliation efficiency.

The effects of detection asymmetry on mutual information, Holevo information, and the asymptotic secure key rate are illustrated in Fig.~\ref{fig:homodyne-all} for homodyne detection and Fig.~\ref{fig:double-homodyne-all} for double homodyne detection.
From Fig.~\ref{fig:homodyne-all}, it is evident that while asymmetry in homodyne detection generally degrades performance, the combination of a non-balanced beamsplitter and unequal detector efficiencies can be beneficial, resulting in maxima of the mutual information and secret fraction with given parameters. Correspondingly, the minima of the Holevo information exhibit similar behavior. % Notably, these functions are symmetric and even with shifting local minima (symmetry point).
Our numerical results for the homodyne case are qualitatively consistent with Ref.~\cite{ruiz2023effects}, namely the dependence of the asymptotic secret fraction on the beam splitter transmittance deviation and balanced detector deviation.
Similar behavior can be seen from Fig.~\ref{fig:double-homodyne-all} for double homodyne. However, while mutual information in double homodyne is generally higher than in homodyne, the significantly greater increase in Holevo information leads to a lower secret key fraction overall. An interesting feature illustrated by the black curves in this figure is that, under ideal detection efficiencies, variations in the beamsplitter transmission cause reduction in the Holevo information. As seen previosly in Fig.~\ref{fig:Entropy}, this is the case of joint entropy being constant in $r$ and therefore $C_S^2$ per Eq.~\eqref{eq:r-ideal}, so that only dependency of conditional entropy on asymmetry remains. Conditional entropy is reduced by asymmetry, see Eq.~\eqref{eq:nu3-dh}.

The dependence of the asymptotic secure key rate on channel length in the presence of asymmetrical detection is shown in Fig.~\ref{fig:r-of-l-all}. These results indicate that in the untrusted noise scenario, asymmetry noise significantly reduces maximal channel length. Consistent with previous results, asymmetrical double homodyne performs worse than asymmetrical homodyne, due to Holevo information being high. The discrepancy in our Holevo information calculations leads to results for the asymptotic secret fraction dependence under double homodyne detection that differ from those in Ref.~\cite{Bartlett:25}, although both works reach the same conclusion regarding its degraded performance.

\begin{figure*}
    \centering
    \begin{subfigure}[c]{.3\linewidth}
\includegraphics[width=\linewidth, ]{pics/untrust/hom/IAB.pdf}
\caption[]{}%{$I_{\ind{AB}}$}
        \end{subfigure}
\hfill
        \begin{subfigure}[c]{.3\linewidth}
 \includegraphics[width=\linewidth, ]{pics/untrust/hom/chi.pdf}
\caption[]{}%{$\chi_{\ind{EB}}$}
\end{subfigure}
\hfill
    \begin{subfigure}[c]{.3\linewidth}
\includegraphics[width=\linewidth, ]{pics/untrust/hom/r.pdf}
\caption[]{}%{$R$}
        \end{subfigure}
        \caption{(a) Mutual information, (b) Holevo information, and (c) asymptotic secret fraction using homodyne measurement as functions of the beam splitter transmission, for various detector efficiencies at $V_{\ind{A}}=1, T=0.95, \xi=0.5\cdot10^{-3}, \beta=0.95$
}
\label{fig:homodyne-all}
\end{figure*}

\begin{figure*}
    \centering
    \begin{subfigure}[c]{.3\linewidth}
\includegraphics[width=\linewidth, ]{pics/untrust/dhom/IAB.pdf}
\caption[]{}%{$I_{\ind{AB}}$}
        \end{subfigure}
\hfill
        \begin{subfigure}[c]{.3\linewidth}
 \includegraphics[width=\linewidth, ]{pics/untrust/dhom/chi.pdf}
\caption[]{}%{$\chi_{\ind{EB}}$}
\end{subfigure}
\hfill
    \begin{subfigure}[c]{.3\linewidth}
\includegraphics[width=\linewidth, ]{pics/untrust/dhom/r.pdf}
\caption[]{}%{$R$}
        \end{subfigure}
        \caption{(a) Mutual information, (b) Holevo information, and (c) asymptotic secret fraction using double homodyne detection as functions of the signal beam splitter transmission. All other beam splitters are assumed to be balanced. Results are shown for various detector efficiencies; efficiencies not specified in the legend are taken to be unity. Parameters: $V_{\ind{A}}=1
        , T=0.95, \xi=0.5\cdot10^{-3}, \beta=0.95$.}

\label{fig:double-homodyne-all}
\end{figure*}
\begin{figure*}
    \centering
    \begin{subfigure}[]{.45\textwidth}
\includegraphics[width=\linewidth]{pics/untrust/hom/r(l).pdf}
\caption[]{Homodyne detection}
\label{fig:hom-r}
        \end{subfigure}
        \hfill
        \begin{subfigure}[]{.45\textwidth}
 \includegraphics[width=\linewidth]{pics/untrust/dhom/r(l).pdf}
\caption[]{Double homodyne detection}
\label{fig:dhom-r}
\end{subfigure}
\caption{Asymptotic secret fraction as a function of channel length computed for (a) homodyne detection and (b) double homodyne detection, shown for various detector efficiencies. Parameters are $V_{\ind{A}}=1,T=0.95,\xi=0.5\cdot10^{-3}, \beta=0.95$. The losses are assumed to be 20 dB per 100 km.}

\label{fig:r-of-l-all}
\end{figure*}

\section{Discussion and conclusion}
\label{sec:conclusion}
%%%%%%%%%%%%%%%%%%%%%%%%%%%%

In this paper,
we have studied the effects of asymmetry
introduced by unbalanced beam splitters and different efficiencies of the photodetectors
in photocount statistics of homodyne and double homodyne detection.
%We have performed numerical analysis to explore the applicability range for the Gaussian approximation derived by approximating the Poisson distributions using the probability density functions of normally distrbuted random variables.

By using the Gaussian approximation,
we have developed
the method for constructing POVMs of homodyne-based
schematics.
This method is applied to deduce
the expression for the POVM
that generalizes the well-known results to the case of the asymmetric homodyne detection.
This POVM is found to be well defined across all  parameter settings of the scheme
with the excess noise variance modified by the asymmetry 
and incorporates the effect of asymmetry-induced shift of the mean value.
%(these effects are illustrated in Fig.~\ref{fig:dist-H}).


%We have used the total variational distance~\eqref{eq:stat-dist}, $D_P$,
%to quantify the statistical distance between the Skellam and Gaussian distributions
%and evaluate the accuracy of the Gaussian approximation across various asymmetry parameters.
%We find that, for the signal mode prepared in the coherent state $\ket{\alpha}$,
%the distance increases with $|\alpha|$ (see Fig.~\ref{fig:amp_05}) and,
%in the small-amplitude region with $|\alpha|\le 0.1$,
%the maximum value of the distance can be estimated at about $0.13$
%reached when the local oscillator amplitude $|\alpha_L|$ is in
%the vicinity of unity. At $|\alpha_L|>1$, the distance rapidly drops with
%the LO amplitude. For example, at $|\alpha|=0.5$,
%the distance falls below $0.05$ when the ratio $|\alpha_L|/|\alpha|$ exceeds five
%(see Fig.~\ref{fig:amp_01}).

%We have found that
%(see Figs.~\ref{fig:delta-eta} and~\ref{fig:delta-theta})
%dependence of the distance on
%the photodetector efficiencies
%(the disbalance angle of the beam splitter)
%is sensitive to the disbalance of the beam splitter
%(photodetector efficiencies).
%Thus,
%by varying parameters of BS we may achieve the best approximation for given
%quantum efficiencies and vice versa.
%In general,
%our findings indicate that the quality of the agreement between
%the exact and approximate photocount distributions varies with
%the degree of asymmetry.


Our formalism allows us to easily analyze various homodyne-based schematics. It can be
extended to describe more complex measurement systems.
We have demonstrated this by performing an analysis for
the eight-port asymmetric  double homodyne scheme.
For this scheme,
we have deduced the Gaussian approximation
(see Eqs.~\eqref{eq:Gaussian-8p}-\eqref{eq:mu_G-i})
and the corresponding POVM expressed in terms of the
projectors onto coherent states (see Eq.~\eqref{eq:povm-hetero}).
As is shown in Fig.~\ref{fig:dh-statistics},
the asymmetry induces effects such as shifts and
anisotropy of the distributions in the photocount difference plane. 

%In contrast to the case of homodyne detection,
%the applicability of
%double homodyne POVM~\eqref{eq:povm-hetero}
%may be broken provided that the beam splitter for the signal mode
%is unbalanced.
%This leads to the effects that deserve a separate publication,
%where we will discuss how to extend our approach to
%this case and related issues.

As shown in the main text, the applicability of double homodyne POVM in the form ~\eqref{eq:povm-hetero} may be broken provided that the beam splitter for the signal mode is unbalanced. To resolve this, an extension to the set of squeezed coherent states is needed (see Eq.~\eqref{eq:Pi_G-gen-x1x2}), leading to explicit dependency of POVM on the squeezing parameter, which is defined by interval~\eqref{eq:r1-r2}, illustrated by Fig.~\ref{fig:squeezing}. This implies that there are, generally, infinitely many POVMs representing one set of the parameters of double homodyne scheme, and so the squeezing parameter needs additional rule to make the POVM well-defined. This becomes relevant in untrusted noise security analysis. Namely, while mutual information between Alice and Bob and conditional entropy in Gaussian modulated coherent states CV-QKD protocol are calculated with variances of quadrature distribution~\eqref{eq:P_G-x1x2} (see Appendix~\ref{sec:MI-and-EBtoPm} for explicit calculations), joint entropy, and thereby Holevo information, is explicitly dependent on the squeezing parameter (see Fig.~\ref{fig:Entropy}), due to the need of separating asymmetry noise from perfect measurement. We optimize the squeezing parameter to maximize Holevo information.

%Note that, according to Appendix~\ref{sec:appendix_comparison},
%an alternative method based on the Gaussian approximation
%for the Bessel function that enters the Skellam distribution
%generally leads to  results that are not suitable for dealing with
%asymmetry-induced effects in homodyne detection.

%\textcolor{red}{Our concluding remark is to put our results in the context of CV-QKD security{~\cite{zhang2024continuous}}. 
%The asymmetry effects described in the paper, such as deviations from the ideal 50:50 beam splitter ratio and mismatched detector efficiencies, introduce vulnerabilities into CV-QKD systems. %These flaws not only allow quantum hackers to compromise security but also degrade the system’s performance~\cite{ruiz2023effects}. 
%These vulnerabilities can be exploited by adversaries through attack strategies such as the wavelength attack \cite{huang2012wavelength, huang2014quantum} and the homodyne detector blinding \cite{qin2018homodyne} and saturation~\cite{qin2016quantum}. The wavelength attack leverages the wavelength-dependent coupling ratio of fiber beam splitters and can be countered by using proper spectral filtering. Blinding and saturation attacks exploit the saturation behavior of homodyne detectors, and their effectiveness is amplified by receiver imbalance. If the splitting ratio deviates from ideal one, an injected bright pulse more easily displaces the detector output, facilitating saturation and biasing excess noise estimation.} 

In the main text, we analyzed how measurement asymmetry impacts the performance of CV-QKD system in the untrusted noise scenario, where asymmetry noise is accessible to Eve. In this case, even relatively small asymmetry significantly reduces maximum channel length (see Fig.~\ref{fig:r-of-l-all}). In trusted-noise CV-QKD~\cite{usenko2016trusted}, if Alice and Bob are unaware of detector’s arms asymmetry, they may misinterpret the increased variance as channel excess noise rather than trusted detector imperfection. This leads to an overestimation of channel excess noise and an underestimation of trusted detector noise. Similar to blinding attacks~\cite{qin2018homodyne}, this vulnerability can be mitigated by inserting attenuators to balance the detection scheme.
Despite the existence of countermeasures for the attacks described, implementing analytical corrections based on the formalism developed in this paper would be preferable for accurate security assessment and performance optimization.

% This method relies on approximating the exact statistics with the Gaussian approximation. We applied this method to
% write POVM for homodyne and double homodyne schemes. For the double homodyne scheme, we have
% encountered a problem with our approach that needs further investigation. Moreover, we have
% studied the applicability of Gaussian approximation in the case of unequal non-unity quantum efficiency
% of photodetectors and unbalanced beam splitters. We have found that the usual approach of
% approximating the Skellam distribution using the asymptotic expansion of the modified Bessel
% function of the first kind is not applicable in the asymmetrical case.


%DeepSeek on homodyne measurements

% Quantum homodyne measurements are a technique used in quantum optics to measure the quadrature
% components of a quantum electromagnetic field.
% These quadrature components represent the amplitude
% and phase of the field, which are crucial for characterizing quantum states of light, such as
% squeezed states or coherent states. 


% ### Key Concepts:

% 1. **Quadrature Components**:
%    - The electric field of a quantum optical mode can be described in terms of two quadrature
%    components, which are analogous to the position and momentum operators in
%    quantum mechanics. 
%    - These components are defined as:
%      \[
%      X = \frac{1}{2}(a + a^\dagger), \quad P = \frac{1}{2i}(a - a^\dagger)
%      \]
%      where \( a \) and \( a^\dagger \) are the annihilation and creation operators, respectively.

% 2. **Homodyne Detection**:
%    - Homodyne detection involves mixing the quantum field of interest with a strong classical
%    reference field (called the local oscillator) at a beam splitter. 
%    - The local oscillator is typically a coherent state with a well-defined phase, which allows the
%    measurement of a specific quadrature of the input field. 
%    - The output beams from the beam splitter are detected by photodetectors, and the difference in
%    the photocurrents is proportional to the quadrature component of the input field that is in phase
%    with the local oscillator. 

% 3. **Phase Sensitivity**:
%    - By adjusting the phase of the local oscillator, different quadrature components of the input
%    field can be measured. For example, if the local oscillator phase is set to zero, the \( X \)
%    quadrature is measured, while a phase shift of \( \pi/2 \) allows measurement of the \( P \)
%    quadrature. 

% 4. **Applications**:
%    - Quantum homodyne measurements are essential for quantum state tomography, where the complete
%    quantum state of a light field is reconstructed from a set of quadrature measurements. 
%    - They are also used in continuous-variable quantum information processing, such as quantum
%    teleportation and quantum key distribution. 

% ### Mathematical Description:

% The homodyne measurement outcome for a quadrature \( X_\theta \) (where \( \theta \) is the phase of the local oscillator) is given by:
% \[
% X_\theta = X \cos \theta + P \sin \theta
% \]


% The probability distribution of the measurement outcomes provides information about the quantum state of the input field.

% ### Summary:

% Quantum homodyne measurements are a powerful tool in quantum optics for probing the quadrature
% components of a quantum electromagnetic field. By using a local oscillator and balanced detection,
% these measurements allow for the precise characterization of quantum states of light, enabling
% advancements in quantum information science and quantum metrology. 


%%%%%%%%%%%%%%%%%%%%%%%%%%%%%%
\section*{Acknowledgements}
The work was supported by the Russian Science Foundation (project No. 24-11-00398).
%%%%%%%%%%%%%%%%%%%%%%%%%%%%%%%%%

\appendix
\section{Statistical distance between Gaussian approximation and Skellam distribution}
\label{sec:appendix_numerical}

\begin{figure*}[!htb]
  \centering
  \begin{subfigure}{0.49\textwidth}
    \includegraphics[width=.9\linewidth]{pics/homodyne/ED = ED(a)_eta.pdf}
    \caption{$|\alpha_L|=5$}
    \label{fig:amp_05}
\end{subfigure}
\hfill
  \begin{subfigure}{0.49\textwidth}
    \includegraphics[width=.9\linewidth]{pics/homodyne/ED = ED(aL)_eta.pdf}
    \caption{$|\alpha|=0.5$}
    \label{fig:amp_01}
\end{subfigure}
\caption{
  Statistical distance $D_P=\mathtt{D}(\prob,\prob_G)$
  as a function of (a)~signal amplitude and (b)~LO amplitude at different
          detector efficiencies.
        }
  \label{fig:amplitude}
\end{figure*}
\begin{figure*}
    \centering
    \begin{subfigure}[b]{.45\linewidth}
\includegraphics[width=\linewidth]{pics/homodyne/ed(eta2).pdf}
\caption[]{$\eta_1=1$}
\label{fig:eta1-1}
        \end{subfigure}
        \hfill
\begin{subfigure}[b]{.45\linewidth}
 \includegraphics[width=.97\linewidth]{pics/homodyne/ed(eta1).pdf}
 \caption[]{$\eta_2=1$}
 \label{fig:eta2-1}
\end{subfigure}
\caption{Statistical distance $D_P=\mathtt{D}(\prob,\prob_G)$
  as a function of  photodetector efficiency
  (a)~$\eta_2$ at $\eta_1=1$ and
  (b)~$\eta_1$ at $\eta_2=1$
  for different values of the beam splitter disbalance angle
  $\delta\theta$ (see eq.~\eqref{eq:delta-theta}).
  The amplitudes are
  $|\alpha|=1$ and $|\alpha_L|=5$.
}
\label{fig:delta-eta}
\end{figure*}


In this Appendix, we study the perfomance of Gaussian approximation for the statistics of photon count difference~\eqref{eq:Gaussian}:
\begin{align}
  &
  \label{eq:App_Gaussian}
    \prob_G(\mu)=G(\mu-\mu_G;\sigma_G),
    \\
  &
    \label{eq:App_sigma_G}
    \sigma_G=\eta_1|\alpha_1|^2+\eta_2|\alpha_2|^2
    \approx
    (\eta_1S^2+\eta_2C^2) |\alpha_L|^2,
  \\
  &
    \label{eq:App_mu_G}
    \mu_G=\eta_1|\alpha_1|^2-\eta_2|\alpha_2|^2
    \approx
    (\eta_1S^2-\eta_2C^2) |\alpha_L|^2
    \notag
  \\
  &
    +CS (\eta_1+\eta_2) |\alpha_L| \avr{\hat{x}_\phi}
\end{align}
by comparing it with the exact statistics of difference events
governed by the Skellam distribution~\eqref{eq:accurate}:
\begin{align}
  &
  \label{eq:App_accurate}
  \prob(\mu)=e^{-\eta_1|\alpha_1|^2}e^{-\eta_2|\alpha_2|^2}
    \Biggl(\frac{\eta_1|\alpha_1|^2}{\eta_2|\alpha_2|^2}\Biggr)^{\mu/2}
    \notag
  \\
  &
    \times
I_{\mu}\bigl(2\sqrt{\eta_1\eta_2|\alpha_1|^2|\alpha_2|^2}\bigr),
\end{align}
both repeated here for ease. We will limit our numerical results to the coherent signal state only, except in cases where the curves show sufficiently distinct differences.

We evaluate the statistical distance between the probability distributions using the total variational distance
that can be computed as half of the $L^1$ distance
\begin{align}
  &
    \label{eq:stat-dist}
    D_P\equiv\mathtt{D}\left(\prob,\prob_{G}\right)\equiv
    \frac{1}{2}
    \sum_{\mu=-\infty}^{\infty}
    \lvert\prob(\mu) -\prob_G(\mu)\rvert.
  \end{align}
  Note that, according to Eq.~\eqref{eq:stat-dist},
  $\mu$ takes integer values and
  we evaluate the distance between the probability mass fuctions,
  whereas the normalization condition for
the Gaussian function~\eqref{eq:App_Gaussian}
\begin{align}
    \label{eq:norm-cont}
  \int_{-\infty}^{\infty}
    \prob_G(\mu)
  \dd\mu
    =1
\end{align}
implies applicability of the continuum limit.
For integer $\mu$,
the integral
on the left hand side of Eq.~\eqref{eq:norm-cont}
should be replaced with a sum
and we have the relation
\begin{align}
  \label{eq:discr-norm}
  \sum_{\mu=-\infty}^{\infty} \prob_G(\mu)=\vartheta_3(\pi\mu_G,e^{-2\pi^2\sigma_G})\equiv N_G
\end{align}
where $\vartheta_3$ is the Jacobi elliptic theta function~\cite{NIST:hndbk:2010}.

In the applicability region of
the continuum limit, the normalization constant $N_G$
is close to unity. The numerical analysis shows that
$|N_G-1|\le 10^{-4}$ at $ 2\sigma_{G}\ge 1$.
The latter gives the condition for the LO amplitude
\begin{equation}
    |\alpha_L|\geq \frac{1}{\sqrt{2(\eta_1S^2+\eta_2C^2)}}\equiv\alpha_N
    \label{eq:renorm}
  \end{equation}
   which ensures both applicability of the continuum limit
  and proper normalization of the Gaussian approximation.
  In our calculations,
the probability $\prob_G$ will be numerically corrected by introducing
the factor $N_G^{-1}$
provided that $|\alpha_L|$ is below the "renormalization point" $\alpha_N$.


  The curves presented in Fig.~\ref{fig:amplitude}
  illustrate how the accuracy of the Gaussian approximation
  is affected by the signal and LO amplitudes, $|\alpha|$ and $|\alpha_L|$.
  More specifically, in
Fig.~\ref{fig:amp_05} (Fig.~\ref{fig:amp_01}),
  the statistical distance is numerically evaluated as a function
  of the amplitude $|\alpha|$ ($|\alpha_L|$)
at different values of the photodetectors efficiencies
provided that the value of
the other amplitude $|\alpha_L|$ ($|\alpha|$) is fixed.

\begin{figure*}
    \centering
    \begin{subfigure}[b]{.45\textwidth}
\includegraphics[width=\linewidth]{pics/homodyne/full.pdf}
\caption[]{$\ket{\psi}=\ket{\alpha},\:\alpha=0.5$}
\label{fig:dth_alp}
        \end{subfigure}
        \hfill
        \begin{subfigure}[b]{.45\textwidth}
 \includegraphics[width=.97\linewidth]{pics/homodyne/full_fock.pdf}
\caption[]{$\ket{\psi}=\ket{n},\:n=1$}
\label{fig:dth_fock}
\end{subfigure}
\caption{
  Statistical distance $D_P=\mathtt{D}(\prob,\prob_G)$
  as a function of the beam splitter disbalance angle, $\delta\theta$,
for the signal mode prepared in (a)~the coherent state and
  in (b)~the single photon Fock state at different efficiencies  with $|\alpha_L|=5$.
    }
    \label{fig:delta-theta}
\end{figure*}


Referring to Fig.~\ref{fig:amp_05},
the curves behave as expected:
given the LO amplitude $|\alpha_L|$,
the distance monotonically increases with $|\alpha|$.
It is shown that,
at $|\alpha_L|=5$ and $|\alpha|>1$,
the perfectly symmetric homodyne presents the case
with minimal distance, $D_P$, while in the presence of asymmetry,
the curves exhibit a rapid growth and
the quality of  the Gaussian approximation
rapidly degrades to the point, where $D_P>0.1$,
so it is not useful for its intended purpose.

When it comes to dependencies of the statistical distance on
the LO oscillator amplitude
computed at fixed value of $|\alpha|$,
the above results suggest that the smaller the amplitude $|\alpha|$
the better the accuracy of the Gaussian approximation.
We can also expect the distance
will be small provided that $|\alpha_L|$ is large
and the strong-LO approximation is applicable. 


Referring to Fig.~\ref{fig:amp_01},
the curves evaluated at $|\alpha|=0.5$
display a non-monotonic behavior with two local maxima
in the weak LO range where $|\alpha_L|<1$.
By contrast, after the second maximum at $|\alpha_L|>1$,
the distance falls with the LO amplitude
and it drops below $0.05$ at $|\alpha_L|>2$.

Note, that, when $|\alpha|<0.1$
and the signal mode state is close to the vacuum state,
the two local maxima of $D_P$ can be estimated to be slightly above $0.08$ and $0.1$,
respectively. So, in this case, the distribution~\eqref{eq:Gaussian}
might be regarded as a reasonable approximation even in the
weak LO range where
the probability $\prob_G$ approaches the close neighborhood of
the singular limit,
$\lim_{|\alpha_L|\to 0}\prob_G(\mu)=\delta(\mu)$. 


From Fig.~\ref{fig:amp_01},
it can also be seen that
the distance vs LO amplitude dependence
that can be used as a tool to characterize
the applicability region of the strong-LO approximation
is nearly insensitive to asymmetry.
In other words, the latter does not produce noticeable effects on
the accuracy of the approximation.


The parameters describing
the photodetection asymmetry are
the efficiencies $\eta_1$ and $\eta_2$.
The curves plotted in Fig.~\ref{fig:amplitude}
are computed at different values of the efficiencies.

To quantify deviation of the beam splitter
transmission and reflection amplitudes
from the balanced $50:50$ values
$t=\cos\theta=r=\sin\theta=1/\sqrt{2}$ at
the angle $\theta=\pi/4$,
we introduce the beam splitter \textit{disbalance angle} given by
\begin{equation}
  \label{eq:delta-theta}
        \delta\theta\equiv\frac{\pi}{4}-\theta.
      \end{equation}

      In Figure~\ref{fig:delta-eta}
      we plot the distance against
      the efficiency of the photodetector assuming that the
      other photodetector is perfect.
      The curves are evaluated at different values of the
      disbalance angle~\eqref{eq:delta-theta}.

In Fig.~\ref{fig:delta-theta} 
the statistical distance vs disbalance angle
curves are presented for coherent and one-photon signal states. 
These curves illustrate how the beam splitter disbalance and
the photodetector efficiencies influence the accuracy of the
Gaussian approximation.
The distance is shown to be minimal in the vicinity of
the balanced beam splitter point with
a vanishing disbalance angle, $\delta\theta=0$.
The efficiency dependence of
the distance
is shown to decrease monotonically at $\delta\theta=0$.
In contrast, for disbalanced beam splitter, this dependence can
reveal non-motonic behavior.

% %%%%%%%%%%%%%%%%%%%%%%%%%%%%%%
\section{Gaussian approximation from Skellam distribution}
\label{sec:appendix_comparison}

%\section{Bessel approximation}%???????
%\label{app:bessel}


The derivation procedure for the Gaussian approximation
outlined in Sec.~\ref{sec:homodyne} transforms
the photocount difference probability~\eqref{eq:poisson}
into the form of a convolution of the normal distributions
by approximating the Poisson distributions.
In this Appendix,
we discuss an alternative method
where the starting point is the Skellam distribution~\eqref{eq:accurate}.
For convenience, we shall reproduce the expression for
this distribution here:
\begin{align}
  &
\label{eq:Skellam}   
  P(\mu)=e^{-\eta_1|\alpha_1|^2}e^{-\eta_2|\alpha_2|^2}
  \left(\frac{\eta_1|\alpha_1|^2}{\eta_2|\alpha_2|^2}\right)^{\frac{\mu}{2}}
  \notag
  \\
  &
  \times
I_{\mu}\left(2\sqrt{\eta_1\eta_2|\alpha_1|^2|\alpha_2|^2}\right),
\end{align}
where $I_k(z)$ is the modified Bessel function of the first kind
and the amplitudes $|\alpha_{1,2}|$ are given by Eq.~\eqref{eq:amplitudes}.

The method under consideration
(see, e.g. the textbook~\cite{Vogel:bk:2006})
assumes that,
in the strong LO limit, the argument of the modified Bessel function is large
and $I_\mu(z)$ can be
approximated using its asymptotic expansion
taken in the Gaussian form:
\begin{align}
  \label{eq:asymp-A}
  I_\mu(z)\approx\frac{1}{\sqrt{2\pi z}}\exp\left[z-\frac{\mu^2}{2 z}\right].
\end{align}
This formula can be deduced by
performing a saddle-point analysis
for the integral representation of
the Bessel functions~\cite{freyberger1993photon}.

Heuristically, it can also be obtained from
from the lowest order asymptotic expansion for
the Bessel function~\cite{NIST:hndbk:2010}:
$I_{\mu}(z)\approx e^z (1-(4\mu^2-1)/(8z))/\sqrt{2\pi z}$
assuming that, for small values of $x$,
$1-x$ can be replaced with $e^{-x}$
(the factors independent of $\mu$ are not essential because they can be incorporated
into the normalization factor of the Gaussian approximation).


\begin{figure}[ht!]
    \centering
    \includegraphics[width=.75\linewidth]{pics/appendix/app_1.pdf}
    \caption{Distances, $D_P=D(\prob,\prob_G)$ and $D_S=D(P,{\prob}_G^{(s)})$, 
      between the Skellam distribution, $P$, and
      two Gaussian approximations, $\prob_G$ (see Eq.~\eqref{eq:Gaussian})
      and  ${\prob}_G^{(s)}$ (see Eq.~\eqref{eq:tPG-mu}), as a function of $|\alpha_L|$
      the beam splitter disbalance angle  at $\delta\theta=15^{\circ}$
$\alpha=1$, and $\eta_1=\eta_2=1$.
      % Note the significant difference (in first decimal places) between the curves, which means that $P_B$
      % becomes incorrect for higher asymmetry
    }
    \label{fig:PGPGs-aL}
\end{figure}


Assuming that $CS\ne 0$ and $|\alpha_L|$ is sufficiently large,
we can use the approximate relations
\begin{align}
  &
    \label{eq:z1-A}
    z=2\sqrt{\eta_1\eta_2}|\alpha_1||\alpha_2|
    \approx
    2CS\sqrt{\eta_1\eta_2}|\alpha_L|^2,
  \\
  &
    \label{eq:z2-A}
    \ln \left(\frac{{\eta_1}|\alpha_1|^2}{{\eta_2}|\alpha_2|^2}\right)^{\frac{\mu}{2}}\approx
    \frac{\mu}{2}\left(
    \ln\frac{\eta_1S^2}{\eta_2C^2}+\frac{\langle\hat{x}_\phi\rangle}{CS|\alpha_L|}\right)
    \end{align}
    to obtain the Gaussian approximation for the Skellam distribution~\eqref{eq:Skellam}
    given by
\begin{align}
  &
  \label{eq:tPG-mu}
        {\prob}_G^{(s)}(\mu)=G(\mu-\tilde{\mu}_G;\tilde{\sigma}_G),
  \\
  &
  \label{eq:tmu_G}
  \tilde{\mu}_G=\sqrt{\eta_1\eta_2}\Bigl[
  CS |\alpha_L|^2\ln\frac{\eta_1S^2}{\eta_2C^2}+
  |\alpha_L|\avr{\hat{x}_{\phi}}
  \Bigr],
  \\
  &
      \label{eq:tsgm_G}
    \tilde{\sigma}_G=
    2CS\sqrt{\eta_1\eta_2}|\alpha_L|^2.
\end{align}



Similar to Eq.~\eqref{eq:Pgood-w-def},
we cast the probabilty~\eqref{eq:tPG-mu}
into the following quadrature form: 
\begin{align}
  &
    \label{eq:tPG-tx}
{\prob}_G^{(s)}(\tilde{x})=\frac{1}{\sqrt{2\pi\tilde{\sigma}_G}}
    \exp \biggl\{-\frac{(\tilde{x}-\avr{\hat{x}_\phi})^2}{2\tilde{\sigma}_x}\biggr\},
  \\
  &
    \label{eq:tld-x}
    \tilde{x}=\frac{\mu}{{\sqrt{\eta_1\eta_2}|\alpha_L|}}-
    CS|\alpha_L|\ln\frac{\eta_1S^2}{\eta_2C^2},
    \\
  &
    \label{eq:tsgm_x}
    \tilde{\sigma}_x=\frac{2CS}{\sqrt{\eta_1\eta_2}},
\end{align}
so that we may follow the line of reasoning
presented in Sec.~\ref{sec:homodyne} to deduce the POVM
\begin{align}
  &
\label{eq:tpovm}    
    \hat{\Pi}_G^{(s)}=\frac{1}{\sqrt{\eta_1\eta_2}|\alpha_L|}
    \notag
  \\
  &
    \times
    \int \dd x' G(x-x'; \tilde{\sigma}_N)|x',\phi\rangle\langle x',\phi|
\end{align}
with the noise variance
\begin{equation}
  \label{eq:tsigm-N}
  \tilde{\sigma}_N=\tilde{\sigma}_x-1,
  \quad
  0 \le \tilde{\sigma}_x\le \tilde{\sigma}_x^{(\max)}=1/\sqrt{\eta_1\eta_2}.
\end{equation}


\begin{figure}[ht!]
    \centering
    \includegraphics[width=.75\linewidth]{pics/appendix/app_2.pdf}
    \caption{Distances, $D_P=D(\prob,\prob_G)$ and $D_S=D(\prob,\prob_G^{(s)})$, 
      between the Skellam distribution, $P$, and
      two Gaussian approximations, $\prob_G$ (see Eq.~\eqref{eq:Gaussian})
      and  ${\prob}_G^{(s)}$ (see Eq.~\eqref{eq:tPG-mu}), as a function of
      the beam splitter disbalance angle $\delta\theta$ at
$\alpha=1$, $\alpha_L=10$ and
      $\eta_1=\eta_2=1$.
      % Note the significant difference (in first decimal places) between the curves, which means that $P_B$
      % becomes incorrect for higher asymmetry
    }
    \label{fig:PGPGs-dth}
\end{figure}


Note that,
in the symmetric
case with $C=S$ and $\eta_1=\eta_2$,
the Gaussian distributions given by
Eqs.~\eqref{eq:Gaussian} and~\eqref{eq:tPG-mu} are equivalent.
This is no longer the case in the presence of asymmetry effects.

Figure~\ref{fig:PGPGs-aL} demonstrates
that, at $\delta\theta\ne 0$,
by contrast to the distance between $P$ and $\prob_G$,
$D_P=\mathtt{D}(\prob,\prob_G)$ which is a monotonically decreasing function of
$|\alpha_L|$,
the distance
between the Skellam distribution and
the approximate distribution~\eqref{eq:tPG-mu},
$D_S=\mathtt{D}(\prob,\prob_G^{(s)})$,
reveals non-monotonic behaviour and
increases with $|\alpha_L|$ at sufficiently large LO amplitudes.
Referring to Fig.~\ref{fig:PGPGs-dth},
disbalance of the beam splitter
has strong detrimental effect on
the accuracy of the approximation~\eqref{eq:tPG-mu}.

What is more important is
that, by contrast the noise excess variance~\eqref{eq:sgm_N}
which is always positive,
the variance~\eqref{eq:tsigm-N} becomes negative
when $2CS\le \sqrt{\eta_1\eta_2}$.
The latter breaks applicability of Eq.~\eqref{eq:tpovm}
giving an ill-posed POVM.

\section{Mutual Information in GG02 protocol and equivalence between Prepare-and-Measure and Entanglement-based schemes}
\label{sec:MI-and-EBtoPm}
In this appendix, we provide detailed derivations of mutual information between Alice and Bob in the GG02 protocol used in the main text. We also show the equivalency between Prepare-and-Measure (PM) and Entanglement-Based (EB) schemes.

Alice prepares an ensemble of coherent states $\ket{\alpha=q+ip}$
with probabilities $p_{\ind{A}}(\alpha)$ distributed according to Gaussian law,
\begin{align}
\label{eq:probabilities}
    p_{\ind{A}}(\alpha)=\frac{1}{\pi V_{\ind{A}}}\exp\left[
    -\frac{|\alpha|^2}{2V_{\ind{A}}}
    \right],
\end{align}
\begin{align}
    \label{eq:ensemble}
    \rho_{\ind{A}}=\int\mathop{\dd^2\alpha} p_{\ind{A}}(\alpha)
    \ket{\alpha}\hspace{-1pt}
    \bra{\alpha}.
\end{align}

After transmission through a Gaussian channel, which attenuates the coherent amplitude by a factor of $\sqrt{T}$, where $T$ is the channel transmission, the state is transformed as $\ket{\alpha}\mapsto\ket{\sqrt{T}\alpha}\equiv\ket{\tilde\alpha=\tilde q +i \tilde p}$.
If independent channel noise is present, the ensemble reads
\begin{align}
\label{eq:rho-tilde}
    &\tilde\rho_{\ind{A}}=\frac{1}{\pi\xi T}\int\mathop{\dd^2\tilde\alpha}p_{\ind{A}}(\tilde\alpha)
    \int\mathop{\dd^2\alpha'}
    \exp\left[
    -\frac{\left|\alpha'\right|^2}{\xi}\right]\notag\times\\&\ket{\tilde\alpha-\alpha'}\hspace{-1pt}\bra{\tilde\alpha-\alpha'},
\end{align}
where $\xi$ is the channel noise variance. Note that variance of $p_{\ind{A}}(\tilde\alpha)$ is $TV_{\ind{A}}$.

Then, Bob performs a measurement described by POVM $\{\hat{\Pi}_x\}$, where the index $x$ corresponds to the measurement outcomes (e.g., quadrature values $q$ and $p$ in homodyne detection). The probability that Bob obtains measurement outcome $x$ is given by the Born rule:
\begin{align}
\label{eq:pB(xA)}
    &p_{\ind{B}}(x,\tilde\alpha) \sim
    p_{\ind{A}}(\tilde\alpha)\int\mathop{\dd^2\alpha'}\exp\left[
    -\frac{\left|\alpha'\right|^2}{\xi}\right]Q_x(\tilde \alpha-\alpha'),
\end{align}
where $Q_x(\tilde \alpha)$ is the $Q$-function of POVM used.

For homodyne detection, Eq.~\eqref{eq:pB(xA)} takes the form given by  Eq.~\eqref{eq:Pgood-w-def}
\begin{align}
\label{eq:pxa}
   p_{\ind{B}}^{\ind{H}}(x,\tilde \alpha) 
  \sim
    \exp\left[
    -\frac{(x-2\tilde q)^2}{2\left(\sigma_x+2\xi\right)}-\frac{\tilde q^2}{2TV_{\ind{A}}}
    \right].
\end{align}
Rewriting exponential's power in Eq.\eqref{eq:pxa} in quadratic form results in
\begin{align}
    \label{eq:sigma-hom-def}
    &-\frac{1}{2}
    \begin{pmatrix}
        \tilde q&x
    \end{pmatrix}
    \begin{pmatrix}
        {TV_{\ind{A}}}&2{TV_{\ind{A}}}\\
        2{TV_{\ind{A}}}&4TV_{\ind{A}}+\sigma_x+2\xi
    \end{pmatrix}^{-1}
    \begin{pmatrix}
        \tilde q\\x
    \end{pmatrix}\notag\\&\equiv 
    -\frac{1}{2}
    \begin{pmatrix}
        \tilde q&x
    \end{pmatrix}{\Sigma^{\ind{H}}}^{-1}
    \begin{pmatrix}
        \tilde q\\x
    \end{pmatrix},
\end{align}
and with covariance matrix $\Sigma^{\ind{H}}$, mutual information between Alice and Bob can be calculated as follows~\cite{SochEtAl2024_StatProofBook}
\begin{align}
\label{eq:IABPM}
    I_{\ind{AB}}^{\ind{H}}=\frac{1}{2}\log \frac{\Sigma^{\ind{H}}_{11}\Sigma^{\ind{H}}_{22}}
    {\det\Sigma^{\ind{H}}}
    %\frac{1}{2}\log \frac{4TV_{\ind{A}}+\sigma_x+2\xi}{\sigma_x+2\xi}
    =\frac{1}{2}\log\left[1+\frac{4TV_{\ind{A}}}{\sigma_x+2\xi}\right],
\end{align}
where $\Sigma^{\ind{H}}_{ii}$ are diagonal elements of $\Sigma^{\ind{H}}$, and all logarithms are base 2.

If double homodyne detection is used, from Eq.~\eqref{eq:pB(xA)} we have (see Eq.~\eqref{eq:P_G-x1x2}):
\begin{align}
  \label{eq:dh-exp}
    &p_{\ind{B}}^{\ind{DH}}(x,\tilde \alpha) \sim\notag\\
    &\exp\left[-\frac{(x_1-\tilde q)^2}{2\left(\frac{\sigma_1}2+\frac\xi2\right)}-\frac{\tilde q^2}{2TV_{\ind{A}}}
    -\frac{(x_2-\tilde p)^2}{2\left(\frac{\sigma_2}2+\frac\xi2\right)}-\frac{\tilde p^2}{2TV_{\ind{A}}}
    \right].
\end{align}
Analogously to Eq.~\eqref{eq:sigma-hom-def}, we obtain covariance matrices as follows:
\begin{align}
  \label{eq:Sigma-DH}
    &\Sigma^{\ind{DH}^{(i)}}=\begin{pmatrix}
        {TV_{\ind{A}}}&{TV_{\ind{A}}}\\
        {TV_{\ind{A}}}&{TV_{\ind{A}}}+\frac{\sigma_i}{2}+\frac\xi2
    \end{pmatrix}, \quad i = 1,2,
\end{align}
from which we obtain mutual information as
\begin{align}
  \label{eq:I_dh}
    &I_{\ind{AB}}^{\ind{DH}}=\frac{1}{2}\sum_{i=1,2}\log \frac{\Sigma^{\ind{DH}^{(i)}}_{11}\Sigma^{\ind{DH}^{(i)}}_{22}}
    {\det\Sigma^{\ind{DH}^{(i)}}}=\notag\\
    &\frac{1}{2}\sum_i\log\left[1+\frac{2TV_{\ind{A}}}
    {\sigma_i+\xi}\right].
\end{align}

Now we move on to the equivalency between PM and EB schemes.
Consider two mode squeezed vacuum state (TMSVS), in ket notation written as{~\cite{Weed:rmp:2012}}
\begin{align}
  \label{eq:psi_AB}
  \ket{\Psi}_{\ind{AB}}=\sqrt{1-\lambda^2}\sum_{n=0}^{\infty}\lambda^n\ket{n,n}_{\ind{AB}},
\end{align}
where $\lambda=\tanh r,\ r$ is the squeezing parameter. Let Alice hold the state $\varsigma_{\ind{AB}}=\ket{\Psi}_{\ind{AB}}\bra{\Psi}_{\ind{AB}}$. In the entanglement-based (EB) protocol, Alice measures one mode using a double measurement, which corresponds to a POVM of coherent state projectors $\left\{\hat\Pi_\beta=\frac{\ket{\beta}\hspace{-1pt}\bra{\beta}}\pi\right\}$, and sends the second mode to Bob. The probability that Alice observes double homodyne outcome $\beta$ is given by the Born rule:
\begin{align}
  \label{eq:p_A_EB}
  p_{\ind{A}}^{\ind{EB}}(\beta)=\Tr \hat{\Pi}_\beta \varsigma_{\ind{A}},
\end{align}
where
\begin{align}
  \label{eq:vars_A}
  \varsigma_{\ind{A}}=\Tr_{\ind{B}}\varsigma_{\ind{AB}}=\left(1-\lambda^2\right)\sum_{n}\lambda^{2n}\ket{n}\hspace{-1pt}\bra{n}.
\end{align}
is the reduced density matrix of Alice's mode.
Substituing Eq.~\eqref{eq:vars_A} into into Eq.~\eqref{eq:p_A_EB} yields
\begin{align}
  \label{eq:p_A_EB-final}
  p_{\ind{A}}^{\ind{EB}}(\beta)=\frac{\left(1-\lambda^2\right)}{\pi}\exp\left[\left(\lambda^2-1\right)|\beta|^2\right].
\end{align}

After Alice obtains outcome $\beta$, second mode before channel transmission reads
\begin{align}
  \label{eq:vars_B_beta}
    \varsigma_{\ind{B}}^\beta=
    \ket{-\lambda\beta^*}\hspace{-1pt}\bra{-\lambda\beta^*}
\end{align}
and the ensemble reads
\begin{align}
  \label{eq:EB-ensemble}
    \varsigma_{\ind{B}}=\int\mathop{\dd^2\beta}
    p_{\ind{A}}^{\ind{EB}}(\beta)
    \ket{-\lambda\beta^*}\hspace{-1pt}\bra{-\lambda\beta^*},
\end{align}
exactly the same as Eq.~\eqref{eq:ensemble} after substituting
\begin{align}
  \label{eq:EB_subs}
    \alpha=-{\lambda}{\beta^*}, \quad
    \frac{1-\lambda^2}{\lambda^2}=\frac{1}{2V_{\ind{A}}}.
\end{align}

It follows that covariance matrix of TMSVS can be used to calculate Holevo information in the PM protocol.


% This implies that $\sigma$ can be negative for certain combinations of $\eta_{1,2}$ and $\theta$,
% which contradicts the form given in Eq.~\eqref{eq:theform}. Therefore, Eq.~\eqref{eq:Pbad} does not
% fully describe the asymmetrical case, and this is a direct consequence of
% Eq.~\eqref{eq:hole}. 

% Numerical results shown in Fig.~\ref{fig:PB_PG}
% also indicate that $P_G^{(s)}$ is a significantly less
% accurate approximation as compared to $P_G$.

% We can pinpoint the cause of such inaccuracy to
% the approximate terms given by
% Eq.~\eqref{eq:asymp-A} and Eq.~\eqref{eq:z2-A}.
% To this end, we consider the ratio
% % We draw this conclusion from the difference in
% % shifting of exact and approximate formulae, which are described by these terms. To further prove
% % this, we will calculate the relation:
% \begin{align}
%   &
%   \label{eq:ratio}
%     R=\frac{\sqrt{2\pi z}I_{\mu}(z)
%   }{
%     \exp\left[z-\frac{\mu^2}{2z}\right]
%     }
%     \notag
%   \\
%   &
%     \times
%     \frac{\left(\frac{{\eta_1}|\alpha_1|^2}{{\eta_2}|\alpha_2|^2}\right)^{\frac{\mu}{2}}}{\exp\left[\frac{\mu}{2}\left(
%     \ln\frac{\eta_1S^2}{\eta_2C^2}+\frac{\avr{\hat{x}_{\phi}}}{CS|\alpha_L|}\right)
%     \right]},
% \end{align}
% where $z\equiv2\sqrt{\eta_1\eta_2|\alpha_1|^2|\alpha_2|^2}$,
% and evaluate it as a function of $\mu$ depending on the disbalance angle of the beam splitter.
% The results are presented in Fig.~\ref{fig:ratio}.
% It is seen that, at high asymmetry, these approximations do not hold.

% \begin{figure}
%     \centering
%     \includegraphics[width=0.75\linewidth]{pics/appendix/relation.pdf}
%     \caption{Ratio given by Eq.~\eqref{eq:ratio} plotted against $\mu$
%       for different values of $\delta\theta$ at $|\alpha_L|=5$,
%       $|\alpha|=0.1$, and $\eta_1=\eta_2=1$.
%       \textbf{Where is the curve with $\delta\theta=0$?}
%       % Significant deviation from unity represents incorrectness of approximations
%       % Eq.~\eqref{eq:hole} and Eq.~\eqref{eq:hole2}.
%     }
%     \label{fig:ratio}
% \end{figure}

% %%%%%%%%%%%%%%%%%%%%%%

%\bibliography{quant,my_papers,refs}
\bibliography{quant,refs,homodyne,math}

\end{document}



%%% Local Variables:
%%% mode: latex
%%% TeX-master: t
%%% End:
